\section{Итерационный метод решения задачи оптимального проектирования}
\label{section:IterationMethod}

В
\cite{article:Problems}
установлено следующее утверждение.
\begin{theorem}
Оптимальное решение $w$ и собственная функция $y_1$, соответствующая собственному значению $\lambda_1[w]$,
образует седловую точку функционала $\Lambda(\cdot, \cdot)$,
т.\,е.
выполняются следующие неравенства:
\[
\Lambda(u, y_1)
\leq
\Lambda(w, y_1)
\leq
\Lambda(w, z),
\]
где
$z \in V$, $u \in \mathcal{U}$.
\end{theorem}
%
%
%
\par
В соответствии с приведенным утверждением оптимальное решение будем искать с помощью следующей итерационной
процедуры:
\[
\begin{gathered}
\Lambda(u_{n + 1}, y_n)
=
\sup \Lambda(u, y_n),
\\
\Lambda(u_n, y_n)
=
\min \Lambda(u_n, y).
\end{gathered}
\]
Сходимость этой процедуры обосновывается по схеме близкой к
\cite{article:Goncharov:Wing}.