\chapter{Теоретическая часть}
\thispagestyle{fancy}
\par
\textcolor{red}{В этой части мы подробнее раскроем теорию.}

\subsection{Обзор предметной области}

% \begin{definition}
%  Классификация текста --- использование математических моделей для определения принадлежности текста, на
основании его содержимого, к конкретному классу из определенного множества.
% \end{definition}
%
% \begin{definition}
%  Мультиклассификатор текстов --- алгоритм или обученная модель, предсказывающие принадлежность каждого
входного текста к одному или нескольким классам, множество классов определено заранее.
% \end{definition}
%
% \begin{definition}
%  Сентимент-анализ (анализ тональности) --- автоматическое выявление в текстах эмоционально окрашенной
лексики и эмоциональной оценки авторов.
% \end{definition}
%
% \begin{definition}
%  Сентимент --- совокупность чувств и взглядов, как основа для действия или суждения; общая эмоциональная
установка.
% \end{definition}

%%% 1. Semina %%%
\par
Сентимент-анализ, как направление компьютерной лингвистики, стал очень популярен последние десятилетие. С
появлением больших данных насыщенных эмоциональной составляющей возникла потребность в их обработке. Компании
начали устраивать соревнования с внушительными призовыми фондами, а исследователи искать лучшую архитектуру
для классификации этих данных. Так в открытом доступе появились большие размеченные датасететы с отзывами,
данными из соцсетей и новостями.

\bigskip\par
Сам термин <<сентимент>> означает совокупность чувств и взглядов, как основа для действия или суждения; общая
эмоциональная установка. Целью сентимент-анализа является выделении этих тональных компонент из текста.
Рассмотрим его применение на разных уровнях \cite{Semina}.

\bigskip\par
Пусть есть целый документ, тогда, как правило, в нем можно выделить один субъект и один объект, а так же
сентимент. Ярким примером такого уровня является отзыв. Здесь автор выступает в качестве субъекта, а предмет
отзыва --- в качестве объекта. Это уровень документа.

\bigskip\par
Если документ более сложный, то можно рассматривать его на уровне предложений. В результате можно определить
эмоциональную окраску всего документа или предложений по отдельности. Зависит от поставленной задачи.

\bigskip\par
Также анализ бывает на уровне аспектов. Смысл его в том, что эмоциональная установка определяется не для
конкретно объекта, а для его отдельных составляющих --- аспектов. Например, для объекта <<компьютер>> можно
выделить аспекты <<производительность>>, <<дизайн>>, <<сборка>> т.д., другими словами, к аспектам относится
все то, к чему могут быть выражены эмоции. Данная задача очень востребована, потому что позволяет точнее
определять отношение автора к объекту, а для некоторых задач это очень важный критерий.

\bigskip\par
Последний вид анализа самый сложный и проводится уровне именованных сущностей (Named Entities). Именованная
сущность --- это абстрактный или физический объект, который может быть обозначен собственным именем. Сама по
себе задача извлечения именованных сущностей (Named Entity Recognition) не из простых, а вкупе с
сентимент-анализом становится действительно комплексным решением.

% https://habr.com/ru/company/mailru/blog/516214/
% https://ieeexplore.ieee.org/document/9117010/references#references

\bigskip\par
Методы сентимент анализа можно также разделить на несколько основных направлений \cite{Semina}:
\bigskip
\begin{itemize}
 \item \textit{метод основанный на правилах (rule-based).} Здесь используются наборы правил классификаци,
составленные экспертом и эмоционально размеченные словари. Определенный класс присваивается в зависимости от
найденных ключевых слов и их использования с другими ключевыми словами. Такой метод достаточно эффективный, но
очень трудоемкий; % ++ добавить статьи + ATEX

 \item \textit{метод основанный на словарях.} Самый простой метод, основанный  на подсчете сентиментных единиц
содержащихся в словарях тональности. Очень сильно зависит от размера словаря и не очень точен в разрыве с
правилами русского языка. Попытка получения сентимента из текста таким способом описана в части ++ этой
работы;

 \item \textit{методы основанные на машинном и глубоком обучении.} (machine learning, deep learning) Наиболее
популярная группа методов в сентимент-анализе, работающих на способности алгоритмов обобщать выделенные из
текста признаки;

 \item \textit{гибридные методы.} Позволяют одновременно использовать несколько подходов. % ++ дописать сюжа
что-нибудь
\end{itemize}

\bigskip\par
Разберем подробнее машинное и глубокое обучение. Суть метода в выделении признаков из текста и последующем их
обобщении с помощью разнообразных алгоритмов. Для выделения признаков используют, как простые алготитмы по
типу мешок слов (Bag of Words) или TF-IDF, так и небольшие нейронные сети для генерирования эмбеддингов,
например, Word2Vec ++, GloVe ++, FastText ++. Существуют и более сложные алгоритмы, которые формируют признаки
на уровне предложений, к ним относятся ELMo ++, BERT (Bidirectional Encoder Representations from Transformers)
++ и др.

\bigskip\par
Чтобы обработать выделенные признаки используют разнообразные алгоритмы. К классическому обучению относятся:
\bigskip
\begin{itemize}
 \item Байессовский классификатор (Naive Bayes classifier);
 \item дерево решений (Decision Tree);
 \item логистическая регрессия (Logistic Regression);
 \item метод опорных векторов (Support Vector Machine).
\end{itemize}

\bigskip\par
Среди алгоритмов глубокого обучения можно выделить:
\begin{itemize}
 \item
\end{itemize}


\subsection{Задача классификации и сентимент-анализа текстов}












































% \subsection{Свободные колебания балки. Модель Бернулли\nobreak-Эйлера}
% \label{section:GoverningProblem}
%
%
%
% \par
% Если на балочную конструкцию наложить деформирующее воздействие,
% приводящее к выходу из состояния равновесия,
% то под действием внутренних сил возникают \textbf{\emph{свободные (или собственные)}} колебания.
%
%
%
% \par
% В настоящей работе
% рассматривается упругая балка модели Бернули-Эйлера.
%
%
%
% Напомним, что в классической теории Бернули-Эйлера не учитывается влияние инерции вращения элемента балки и
% деформаций поперечного сдвига.
%
%
%
% При этом в рамках данных допущений предполагается, что размеры поперечных сечений малы по сравнению с длиной
% балки.
%
%
%
% Вместе с тем расчеты, основанные на данной модели балки, все же оказываются довольно точными.
%
%
%
% \par
% Уравнение поперечных свободных колебаний балки имеет следующий вид
% \cite{book:Timoshenko}:
% \begin{equation}
% \label{BeamVibrating:Dynamic}
% \frac{\partial^2}{\partial x^2} \left\{EI\frac{\partial^2 z(x,t)}{\partial x^2} \right\}
% +
% \rho S \frac{\partial^2 z(x,t)}{\partial t^2}
% =
% 0,
% \qquad
% x \in (0,l).
% \end{equation}
%
% В уравнении \eqref{BeamVibrating:Dynamic}
% $x$~--- пространственная координата,
% $t$~--- время,
% $\rho$~--- удельная плотность материала,
% $E$~--- модуль Юнга материала балки,
% $S$~--- распределение площади поперечного сечения балки,
% $I$~--- геометрический момент инерции поперечного сечения,
% $z(x,t)$~--- функция прогибов при колебаниях в точке $(x, t)$,
% $l$~--- длина балки.
%
%
%
% \par
% В модели Бернулли-Эйлера
% при свободных поперечных колебаниях функция прогибов меняется во времени по гармоническому закону
% \cite{book:Timoshenko}
% \begin{equation}
% \label{HarmonicLaw}
% z_k(x, t) = y_k(x) e^{i \omega_k t}.
% \end{equation}
%
%
%
% Здесь индекс $k$ обозначает номер формы колебаний,
% $\omega_k$~--- собственные частоты, $y_k$~--- распределение прогибов.
%
%
%
% \begin{definition}
% Наименьшая частота собственных колебаний $\omega_1$ называется
% \textbf{\emph{фундаментальной частотой}} колебаний.
% \end{definition}
%
%
%
% \par
% Подставляя выражение
% \eqref{HarmonicLaw} в уравнение
% \eqref{BeamVibrating:Dynamic},
% получаем основное уравнение
% \begin{equation}
% \label{BeamVibrating:Dimensional}
% \frac{d^2}{dx^2} \left\{EI\frac{d^2 y}{dx^2} \right\} = \omega^2 \rho S y,
% \qquad
% x \in (0, l).
% \end{equation}
%
%
%
% Здесь ради удобства опущен индекс $k$.
%
%
%
% \par
% Кроме того, на функцию прогибов
% $y$
% налагается система ограничений в виде краевых условий,
% которая определяет тип закрепления балки.
%
%
%
% Краевые условия, возникающие в точке $x_0 \in \left\{0,l\right\}$, соответствующие опертому, защемленному
или
% свободному типу крепления балки, в указанном порядке имеют следующий вид:
% \[
% \begin{aligned}
% y(x_0) &= (EIy'')(x_0) = 0,
% \\
% y(x_0) &= y'(x_0) = 0,
% \\
% (EI y'')(x_0) &= \left(EI y''\right)'(x_0) = 0.
% \end{aligned}
% \]
%
%
%
% \par
% \textcolor{red}{
% В дальнейшем будем предполагать, что момент инерции сечения связан с распределением его по площади по закону
% \( I(x) = A_{\alpha_1}S^{\alpha_1}(x),\) где \({\alpha_1}, A_{\alpha_1}\) некоторые постоянные. Безусловно,
% особенный практический интерес представляют значения 1,2 и 3 параметра \(\alpha_1\).}
%
%
%
% \par
% Введем безразмерные переменные
% \begin{gather}
% x'=\frac{x}{l}, \;\;\; u'=\frac{u}{l}, \;\;\; p'=\frac{lS}{V},
% \notag
% \\
% \label{ChangeOfVariables2}
% \lambda'=\frac{\omega^2\rho V}{B'_{\alpha_1}}, \;\;\;
% B'_{\alpha_1}=\frac{EA_{\alpha_1}}{l^3}\left[\frac{V}{l}\right]^{\alpha_1}
% \end{gather}
% и перепишем уравнение \eqref{BeamVibrating:Dimensional}, опуская в дальнейшем штрихи:
% \begin{equation}
% \label{BeamVibrating:ClassicalSetting}
% \left(e u^\nu y''\right)'' = \lambda \rho u y,
% \qquad
% x \in I \triangleq (0, 1).
% \end{equation}
%
%
%
% \par
% В дальнейшем будем предпологать, что балка может быть закреплена одним из перечисленных ниже способов:
% \begin{enumerate}
%
%
%
% \item
% оперта на обоих концах (Рис.~\ref{fig:HingedBeam}):
% \begin{equation}
% y(0) = (e u^\nu y'')(0) = 0,
% \qquad
% y(1) = (e u^\nu y'')(1) = 0;
% \tag{BC${}_0^0$}
% \end{equation}
%
%
%
% \item
% жестко защемлена на обоих концах (Рис.~\ref{fig:ClampedBeam}):
% \begin{equation}
% y(0) = y'(0) = 0,
% \qquad
% y(1) = y'(1) = 0;
% \tag{BC${}_1^1$}
% \end{equation}
%
%
%
%
%
% \item
% жестко защемлена на одном из концов и свободна на другом (консольная балка, Рис.~\ref{fig:ClampedFreeBeam}):
%
%
%
% \begin{equation}
% y(0) = y'(0) = 0,
% \qquad
% (e u^\nu y'')(1) = (e u^\nu y'')'(1) = 0;
% \tag{BC${}_1^2$}
% \end{equation}
%
%
%
% \item
% жестко защемлена на одном из концов и шарнирно оперта на другом
% (Рис.~\ref{fig:ClampedHingedBeam}):
%
%
%
% \begin{equation}
% y(0) = y'(0) = 0,
% \qquad
% y(1) = (e u^\nu y'')(1) = 0.
% \tag{BC${}_1^0$}
% \end{equation}
%
%
%
%
%
% \end{enumerate}
%
%
%
%
%
% \begin{figure}[h]
% \label{fig:HingedBeam}
% 				\centering
% 				\begin{picture}(230,80)(0,0)
% 					\put( 15,15){\line( 1, 0){30}}
% 					\put( 15,15){\line( 1, 2){15}}
% 					\put( 45,15){\line( -1, 2){15}}
% 					\put( 185,15){\line( 1, 0){30}}
% 					\put( 185,15){\line( 1, 2){15}}
% 					\put( 215,15){\line( -1, 2){15}}
% 					\put( 5,45){\line( 1, 0){220}}
% 					\put( 5,75){\line( 1, 0){220}}
% 					\put( 5,45){\line( 0, 1){30}}
% 					\put( 225,45){\line( 0, 1){30}}
% 				\end{picture}
% 				\caption{Опертая на обоих концах балка.}
% \end{figure}
%
%
%
%
%
% \begin{figure}[h]
% \centering
% \begin{picture}(330,80)(0,0)
% \put( 55,35){\line( 1, 0){220}}
% 				\put( 55,65){\line( 1, 0){220}}
% 				\put( 55,15){\line( 0, 1){70}}
% 				\put( 275,15){\line( 0, 1){70}}
% 				\put( 35,15){\line( 5, 6){20}}
% 				\put( 35,35){\line( 5, 6){20}}
% 				\put( 35,55){\line( 5, 6){20}}
% 				\put( 275,25){\line( 5, 6){20}}
% 				\put( 275,45){\line( 5, 6){20}}
% 				\put( 275,65){\line( 5, 6){20}}
% 				\end{picture}
% 				\caption{Жестко защемленная на обоих концах балка.}
% \label{fig:ClampedBeam}
% \end{figure}
%
%
%
%
%
% \begin{figure}[h]
%     \centering
%     \begin{picture}(330,80)(0,0)
%     \put( 55,35){\line( 1, 0){220}}
%     \put( 55,65){\line( 1, 0){220}}
%     \put( 275,35){\line( 0, 1){30}}
%     \put( 55,15){\line( 0, 1){70}}
%     \put( 35,15){\line( 5, 6){20}}
%     \put( 35,35){\line( 5, 6){20}}
%     \put( 35,55){\line( 5, 6){20}}
%     \end{picture}
%     \caption{Консольная балка.}
%     \label{fig:ClampedFreeBeam}
% \end{figure}
%
%
%
%
%
%
%
%
%
%
% \begin{figure}[h]
%     \centering
%     \begin{picture}(330,80)(0,0)
%     \put( 245,05){\line( 1, 0){30}}
%     \put( 245,05){\line( 1, 2){15}}
%     \put( 275,05){\line( -1, 2){15}}
%     \put( 55,35){\line( 1, 0){220}}
%     \put( 55,65){\line( 1, 0){220}}
%     \put( 275,35){\line( 0, 1){30}}
%     \put( 55,15){\line( 0, 1){70}}
%     \put( 35,15){\line( 5, 6){20}}
%     \put( 35,35){\line( 5, 6){20}}
%     \put( 35,55){\line( 5, 6){20}}
%     \end{picture}
%     \caption{Защемленная на левом конце и опертая на правом балка.}
%     \label{fig:ClampedHingedBeam}
% \end{figure}
%
%
%
% \par
% Уравнение
% \eqref{BeamVibrating:ClassicalSetting}
% вместе с краевыми условиями
% $(\mathrm{BC}_i^j)$
% образует \textbf{\emph{краевую задачу на собственные значения}},
% которая состоит в определении пары $(\lambda, y)$,
% для которой справедливо уравнение
% \eqref{BeamVibrating:ClassicalSetting} и выполняются краевые условия $(\mathrm{BC}_i^j)$.
%
%
%
% Важно, что в паре $(\lambda, y)$
% функция $y$ должна быть нетривиальной, т.\,е. не равной тождественно нулю,
% т.\,к. в противном случае уравнение
% \eqref{BeamVibrating:ClassicalSetting} справедливо для любого $\lambda$ в силу его линейности.
%
%
%
% При соблюдении этой договоренности число $\lambda$
% называется \textbf{\emph{собственным значением}},
% а нетривиальное решение $y$ уравнения
% \eqref{BeamVibrating:ClassicalSetting}~--% \textbf{\emph{собственной функцией (элементом)}}.
%
%
%
% Как следует из \S\,\ref{section:WeakSetting},
% задача
% \eqref{BeamVibrating:ClassicalSetting},
% $(\mathrm{BC}_i^j)$
% имеет счетное множество собственных чисел $\{ \lambda_k \}_{k \in \mathbb{N}}$,
% каждое из которых является положительным.
%
%
%
% В общей ситуации для краевой задачи на собственные значения
% собственному числу $\lambda$ может отвечать несколько линейно независимых
% собственных функций.
%
%
%
% Cовокупность всех собственных функций,
% отвечающих данному собственному числу $\lambda$,
% образует линейное пространство.
%
%
%
% Размерность этого пространства называется \textbf{\emph{кратностью}} собственного значения.
%
%
%
% Если кратность равна $1$, то собственное значение
% называется
% \textbf{\emph{простым}}.
%
%
%
% Из \S\,\ref{section:WeakSetting} следует,
% что
% все собственные значения $\lambda_k$ задачи
% \eqref{BeamVibrating:ClassicalSetting},
% $(\mathrm{BC}_i^j)$
% являются простыми.
%
%
%
%
%
% \par
% Собственные частоты $\omega_k$, где $k = 1, 2, \ldots$ образуют
% \textbf{\emph{спектр колебаний}} \cite{book:Timoshenko,book:Banichuk}.
%
%
%
% Если частоты прикладываемых к балке внешних возмущений находятся в интервале
% $(0, \omega_1)$,
% то не возникает негативных резонансных явлений.
%
%
%
% Следовательно,
% целесообразно
% расширить безрезонансный интервал частот $(0, \omega_1)$,
% чтобы обезопасить конструкцию от разрушения.
%
%
%
% Это приводит к задаче максимизации фундаментальной частоты $\omega_1$ свободных колебаний балки
% или соответственно первого собственного значения $\lambda_1$,
% так как в силу
% \eqref{ChangeOfVariables2}
% \[
% \lambda_k =
% \mathcal{O}(\omega_k^2).
% \]
%
%
%
% Решение этой задачи и является целью данной работы.
%
%
%
% Ее постановка рассматривается в
% \S\,\ref{section:OptimalControlProblem}.


















% В настоящей главе рассматривается задача о свободных колебаниях балки модели Бернулли-Эйлера
% и постановка задачи оптимального проектирования,
% состоящая в определении формы балки,
% максимизирующей фундаментальную частоту колебаний.
%
%
%
% В \ref{section:GoverningProblem} выписывается основное дифференциальное уравнение,
% а также краевые условия,
% соответствующие заданным типам закрепления балки.
%
%
%
% В \ref{section:WeakSetting} приводятся важные свойства краевой задачи на собственные значения:
% обсуждается дискретность спектра, кратность собственных значений и знакоопределенность
% собственной функции, отвечающей наименьшему собственному значению.
%
%
%
% Далее, \,\ref{section:ApproximationsMethod} посвящен методу последовательных приближений,
% позволяющему найти приближенное решение рассматриваемой краевой задачи на собственные значения.
%
%
%
% В \,\ref{section:OptimalControlProblem}
% приводится постановка задачи оптимального проектирования
% и обсуждается вопрос о существовании оптимальных решений.
%
%
%
% Наконец,
% в
% \,\ref{section:IterationMethod}
% формулируется итерационная процедура решения поставленной
% задачи оптимального проектирования и обсуждается вопрос о ее сходимости.
%
% \section{Свободные колебания балки. Модель Бернулли\nobreak-Эйлера}
\label{section:GoverningProblem}
%
%
%
\par
Если на балочную конструкцию наложить деформирующее воздействие,
приводящее к выходу из состояния равновесия,
то под действием внутренних сил возникают \textbf{\emph{свободные (или собственные)}} колебания.
%
%
%
\par
В настоящей работе
рассматривается упругая балка модели Бернули-Эйлера.
%
%
%
Напомним, что в классической теории Бернули-Эйлера не учитывается влияние инерции вращения элемента балки и деформаций поперечного сдвига.
%
%
%
При этом в рамках данных допущений предполагается, что размеры поперечных сечений малы по сравнению с длиной балки.
%
%
%
Вместе с тем расчеты, основанные на данной модели балки, все же оказываются довольно точными.
%
%
%
\par
Уравнение поперечных свободных колебаний балки имеет следующий вид
\cite{book:Timoshenko}:
\begin{equation}
\label{BeamVibrating:Dynamic}
\frac{\partial^2}{\partial x^2} \left\{EI\frac{\partial^2 z(x,t)}{\partial x^2} \right\}
+
\rho S \frac{\partial^2 z(x,t)}{\partial t^2}
=
0,
\qquad
x \in (0,l). 
\end{equation}

В уравнении \eqref{BeamVibrating:Dynamic}
$x$~--- пространственная координата,
$t$~--- время,
$\rho$~--- удельная плотность материала,
$E$~--- модуль Юнга материала балки,
$S$~--- распределение площади поперечного сечения балки,
$I$~--- геометрический момент инерции поперечного сечения,
$z(x,t)$~--- функция прогибов при колебаниях в точке $(x, t)$,
$l$~--- длина балки.
%
%
%
\par
В модели Бернулли-Эйлера
при свободных поперечных колебаниях функция прогибов меняется во времени по гармоническому закону
\cite{book:Timoshenko}
\begin{equation}
\label{HarmonicLaw}
z_k(x, t) = y_k(x) e^{i \omega_k t}.
\end{equation}
%
%
%
Здесь индекс $k$ обозначает номер формы колебаний,
$\omega_k$~--- собственные частоты, $y_k$~--- распределение прогибов.
%
%
%
\begin{definition}
Наименьшая частота собственных колебаний $\omega_1$ называется
\textbf{\emph{фундаментальной частотой}} колебаний.
\end{definition}
%
%
%
\par
Подставляя выражение
\eqref{HarmonicLaw} в уравнение
\eqref{BeamVibrating:Dynamic},
получаем основное уравнение
\begin{equation}
\label{BeamVibrating:Dimensional}
\frac{d^2}{dx^2} \left\{EI\frac{d^2 y}{dx^2} \right\} = \omega^2 \rho S y,
\qquad
x \in (0, l).
\end{equation}
%
%
%
Здесь ради удобства опущен индекс $k$.
%
%
%
\par
Кроме того, на функцию прогибов
$y$
налагается система ограничений в виде краевых условий,
которая определяет тип закрепления балки.
%
%
%
Краевые условия, возникающие в точке $x_0 \in \left\{0,l\right\}$, соответствующие опертому, защемленному или свободному типу крепления балки, в указанном порядке имеют следующий вид:
\[
\begin{aligned}
y(x_0) &= (EIy'')(x_0) = 0,
\\
y(x_0) &= y'(x_0) = 0,
\\
(EI y'')(x_0) &= \left(EI y''\right)'(x_0) = 0.
\end{aligned}
\]
%
%
%
\par
\textcolor{red}{
В дальнейшем будем предполагать, что момент инерции сечения связан с распределением его по площади по закону 
\( I(x) = A_{\alpha_1}S^{\alpha_1}(x),\) где \({\alpha_1}, A_{\alpha_1}\) некоторые постоянные. Безусловно, особенный практический интерес представляют значения 1,2 и 3 параметра \(\alpha_1\).}
%
%
%
\par
Введем безразмерные переменные
\begin{gather}
x'=\frac{x}{l}, \;\;\; u'=\frac{u}{l}, \;\;\; p'=\frac{lS}{V},
\notag
\\
\label{ChangeOfVariables2}
\lambda'=\frac{\omega^2\rho V}{B'_{\alpha_1}}, \;\;\; B'_{\alpha_1}=\frac{EA_{\alpha_1}}{l^3}\left[\frac{V}{l}\right]^{\alpha_1}
\end{gather}
и перепишем уравнение \eqref{BeamVibrating:Dimensional}, опуская в дальнейшем штрихи:
\begin{equation}
\label{BeamVibrating:ClassicalSetting}
\left(e u^\nu y''\right)'' = \lambda \rho u y,
\qquad
x \in I \triangleq (0, 1).
\end{equation}
%
%
%
\par
В дальнейшем будем предпологать, что балка может быть закреплена одним из перечисленных ниже способов: 
\begin{enumerate}
%
%
%
\item
оперта на обоих концах (Рис.~\ref{fig:HingedBeam}):
\begin{equation}
y(0) = (e u^\nu y'')(0) = 0,
\qquad
y(1) = (e u^\nu y'')(1) = 0;
\tag{BC${}_0^0$}
\end{equation}
%
%
%
\item
жестко защемлена на обоих концах (Рис.~\ref{fig:ClampedBeam}):
\begin{equation}
y(0) = y'(0) = 0,
\qquad
y(1) = y'(1) = 0;
\tag{BC${}_1^1$}
\end{equation}
%
%
%
%
%
\item
жестко защемлена на одном из концов и свободна на другом (консольная балка, Рис.~\ref{fig:ClampedFreeBeam}):
%
%
%
\begin{equation}
y(0) = y'(0) = 0,
\qquad
(e u^\nu y'')(1) = (e u^\nu y'')'(1) = 0;
\tag{BC${}_1^2$}
\end{equation}
%
%
%
\item
жестко защемлена на одном из концов и шарнирно оперта на другом
(Рис.~\ref{fig:ClampedHingedBeam}):
%
%
%
\begin{equation}
y(0) = y'(0) = 0,
\qquad
y(1) = (e u^\nu y'')(1) = 0.
\tag{BC${}_1^0$}
\end{equation}
%
%
%
%
%
\end{enumerate}
%
%
%
%
%
\begin{figure}[h]
\label{fig:HingedBeam}
				\centering
				\begin{picture}(230,80)(0,0)
					\put( 15,15){\line( 1, 0){30}}
					\put( 15,15){\line( 1, 2){15}}
					\put( 45,15){\line( -1, 2){15}}
					\put( 185,15){\line( 1, 0){30}}
					\put( 185,15){\line( 1, 2){15}}
					\put( 215,15){\line( -1, 2){15}}
					\put( 5,45){\line( 1, 0){220}}
					\put( 5,75){\line( 1, 0){220}}
					\put( 5,45){\line( 0, 1){30}}
					\put( 225,45){\line( 0, 1){30}}
				\end{picture}
				\caption{Опертая на обоих концах балка.}
\end{figure}
%
%
%
%
%
\begin{figure}[h]
\centering
\begin{picture}(330,80)(0,0)
\put( 55,35){\line( 1, 0){220}}
				\put( 55,65){\line( 1, 0){220}}
				\put( 55,15){\line( 0, 1){70}}
				\put( 275,15){\line( 0, 1){70}}
				\put( 35,15){\line( 5, 6){20}}
				\put( 35,35){\line( 5, 6){20}}
				\put( 35,55){\line( 5, 6){20}}
				\put( 275,25){\line( 5, 6){20}}
				\put( 275,45){\line( 5, 6){20}}
				\put( 275,65){\line( 5, 6){20}}
				\end{picture}
				\caption{Жестко защемленная на обоих концах балка.}
\label{fig:ClampedBeam}				
\end{figure}
%
%
%
%
%
\begin{figure}[h]
    \centering
    \begin{picture}(330,80)(0,0)
    \put( 55,35){\line( 1, 0){220}}
    \put( 55,65){\line( 1, 0){220}}
    \put( 275,35){\line( 0, 1){30}}
    \put( 55,15){\line( 0, 1){70}}
    \put( 35,15){\line( 5, 6){20}}
    \put( 35,35){\line( 5, 6){20}}
    \put( 35,55){\line( 5, 6){20}}
    \end{picture}
    \caption{Консольная балка.}
    \label{fig:ClampedFreeBeam}
\end{figure}
%
%
%
%
%
%
%
%
%
%
\begin{figure}[h]
    \centering
    \begin{picture}(330,80)(0,0)
    \put( 245,05){\line( 1, 0){30}}
    \put( 245,05){\line( 1, 2){15}}
    \put( 275,05){\line( -1, 2){15}}
    \put( 55,35){\line( 1, 0){220}}
    \put( 55,65){\line( 1, 0){220}}
    \put( 275,35){\line( 0, 1){30}}
    \put( 55,15){\line( 0, 1){70}}
    \put( 35,15){\line( 5, 6){20}}
    \put( 35,35){\line( 5, 6){20}}
    \put( 35,55){\line( 5, 6){20}}
    \end{picture}
    \caption{Защемленная на левом конце и опертая на правом балка.}
    \label{fig:ClampedHingedBeam}
\end{figure}
%
%
%
\par
Уравнение
\eqref{BeamVibrating:ClassicalSetting}
вместе с краевыми условиями
$(\mathrm{BC}_i^j)$
образует \textbf{\emph{краевую задачу на собственные значения}},
которая состоит в определении пары $(\lambda, y)$,
для которой справедливо уравнение
\eqref{BeamVibrating:ClassicalSetting} и выполняются краевые условия $(\mathrm{BC}_i^j)$.
%
%
%
Важно, что в паре $(\lambda, y)$
функция $y$ должна быть нетривиальной, т.\,е. не равной тождественно нулю,
т.\,к. в противном случае уравнение
\eqref{BeamVibrating:ClassicalSetting} справедливо для любого $\lambda$ в силу его линейности.
%
%
%
При соблюдении этой договоренности число $\lambda$
называется \textbf{\emph{собственным значением}},
а нетривиальное решение $y$ уравнения
\eqref{BeamVibrating:ClassicalSetting}~---
\textbf{\emph{собственной функцией (элементом)}}.
%
%
%
Как следует из \S\,\ref{section:WeakSetting},
задача
\eqref{BeamVibrating:ClassicalSetting},
$(\mathrm{BC}_i^j)$
имеет счетное множество собственных чисел $\{ \lambda_k \}_{k \in \mathbb{N}}$,
каждое из которых является положительным.
%
%
%
В общей ситуации для краевой задачи на собственные значения
собственному числу $\lambda$ может отвечать несколько линейно независимых
собственных функций.
%
%
%
Cовокупность всех собственных функций,
отвечающих данному собственному числу $\lambda$,
образует линейное пространство.
%
%
%
Размерность этого пространства называется \textbf{\emph{кратностью}} собственного значения.
%
%
%
Если кратность равна $1$, то собственное значение
называется
\textbf{\emph{простым}}.
%
%
%
Из \S\,\ref{section:WeakSetting} следует,
что
все собственные значения $\lambda_k$ задачи
\eqref{BeamVibrating:ClassicalSetting},
$(\mathrm{BC}_i^j)$
являются простыми.
%
%
%
%
%
\par
Собственные частоты $\omega_k$, где $k = 1, 2, \ldots$ образуют
\textbf{\emph{спектр колебаний}} \cite{book:Timoshenko,book:Banichuk}.
%
%
%
Если частоты прикладываемых к балке внешних возмущений находятся в интервале
$(0, \omega_1)$,
то не возникает негативных резонансных явлений.
%
%
%
Следовательно,
целесообразно
расширить безрезонансный интервал частот $(0, \omega_1)$,
чтобы обезопасить конструкцию от разрушения.
%
%
%
Это приводит к задаче максимизации фундаментальной частоты $\omega_1$ свободных колебаний балки
или соответственно первого собственного значения $\lambda_1$,
так как в силу
\eqref{ChangeOfVariables2}
\[
\lambda_k =
\mathcal{O}(\omega_k^2).
\]
%
%
%
Решение этой задачи и является целью данной работы.
%
%
%
Ее постановка рассматривается в
\S\,\ref{section:OptimalControlProblem}.

% \section{Обобщенная постановка краевой задачи на собственные значения. Свойства спектра}
\label{section:WeakSetting}

\par
В дальнейших рассмотрениях будем предполагать,
что
допустимое распределение площади поперечных сечений $u$ является элементом 
множества
\[
U = 
\left\{
a \in L^\infty(I) : \;
\alpha \leq a(x) \leq \beta
\right\}.
\]
Здесь $L^\infty(I)$ обозначает пространство всех измеримых существенно ограниченных функций,
определенных на отрезке $[0,1]$,
а параметры $\alpha$ и $\beta$ задают ограничение на возможные значения площади поперечного сечения вдоль пролета.
%
%
%
Предполагается, что
$0 < \alpha < \beta < +\infty$.
%
%
%
Также будем считать,
что
$e, \rho \in L^\infty(I)$ и, кроме того, справедливы неравенства
\[
e(x) \geq e_0,
\qquad
\rho(x) \geq \rho_0,
\]
где $e_0, \rho_0$~--- некоторые положительные постоянные.
%
%
%
\par
Обобщенная постановка краевой задачи на собственные значения
\eqref{BeamVibrating:ClassicalSetting}, $(\mathrm{BC}_i^j)$ имеет следующий вид:
найти пару $(\lambda, y) \in \mathbb{R} \times (V \setminus \{ \vartheta \})$ такую, что
\begin{equation}
\label{BeamVibrating:WeakSetting}
\int_0^1 eu^\nu y'' z'' \, dx
=
\lambda
\int_0^1 \rho u y z \, dx,
\qquad
z \in V.
\end{equation}
%
%
%
Здесь $V = V_i^j$ обозначает некоторое подпространство пространства Соболева
(конкретный вид которого будет представлен далее),
называемое \textbf{\emph{пространством состояний}},
отвечающее краевым условиям $(\mathrm{BC}_i^j)$,
которые, в свою очередь, соответствуют заданному типу крепления балки,
а
$\vartheta$~--- нулевой элемент этого пространства.
%
%
%
Заметим,
что
\eqref{BeamVibrating:WeakSetting} представляет собой интегральное тождество,
справедливое для всех элементов $z$ пространства $V$.
%
%
%
\par
Введем понятие пространства Соболева
\cite{book:OneDimVarProblems}.
%
%
%
Пусть $C^2([0, 1])$ обозначает пространство всех дважды непрерывно дифференцируемых функций,
определенных на отрезке $[0, 1]$.
%
%
%
Для элемента $y$ этого пространства введем норму по формуле
\begin{equation}
\label{SobolevNorm}
\lVert
y
\rVert_{H^2(I)}
=
\left(\int_0^1 \left( |y|^2 + |y'|^2 + |y''|^2 \right) \, dx\right)^{1/2}.
\end{equation}
%
%
%
\begin{definition}
Замыкание $C^2([0, 1])$ по норме
\eqref{SobolevNorm}
обозначается через $H^2(I)$ и называется \emph{\textbf{пространством Соболева}}.
\end{definition}
%
%
%
Как следует из
\cite{book:Fichera},
реализации пространства $V$,
отвечающие рассматриваемым типам закрепления балки,
имеют следующий вид:
\[
\begin{aligned}
V_0^0 &= \left\{
z \in H^2(I) : \;
z(0) = z(1) = 0
\right\},
\\
V_0^1 &= \left\{
z \in H^2(I) : \;
z(0) = z(1) = z'(1) = 0
\right\},
\\
V_1^1 &= \left\{
z \in H^2(I) : \;
z(0) = z'(0) = z(1) = z'(1) = 0
\right\},
\\
V_1^2 &= \left\{
z \in H^2(I) : \;
z(0) = z'(0) = 0
\right\}.
\end{aligned}
\]
%
%
%
\par
Через $L^2(I)$ будем обозначать пространство Лебега, состоящее из измеримых функций $y$,
квадраты которых являются суммируемыми,
т.\,е.
для которых
конечен следующий интеграл
\[
\int_0^1 y^2 \, dx.
\]
Напомним,
что норма элемента $y$ в пространстве $L^2(I)$
определяется формулой
\[
\lVert y \rVert_{L^2(I)} = \sqrt{\int_0^1 y^2 \, dx}.
\]
%
%
%
\par
Определим билинейные формы 
\[
\begin{aligned}
\mathcal{A}_u &: V \times V \to \mathbb{R},
\\
\mathcal{B}_u &: L^2(I) \times L^2(I) \to \mathbb{R},
\end{aligned}
\]
задаваемые формулами
\[
\begin{aligned}
\mathcal{A}_u(y, z) &= \int_0^1 eu^\nu y'' z'' \, dx,
\\
\mathcal{B}_u(y, z) &= \int_0^1 \rho u y z \, dx.
\end{aligned}
\]
%
%
%
Используя введенные обозначения,
перепишем
\eqref{BeamVibrating:WeakSetting}
в следующем виде:
\[
\mathcal{A}_u(y, z) = \lambda \mathcal{B}_u(y, z),
\qquad
z \in V.
\]
%
%
%
Для того чтобы выяснить вопрос о существовании собственных значений задачи \eqref{BeamVibrating:WeakSetting},
необходимо исследовать свойства введенных билинейных форм.
%
%
%
Пусть $H$~--- пространство Гильберта, $C : H \times H \to \mathbb{R}$~--- билинейная форма,
а
$\lVert \cdot \rVert_H$ обозначает норму в пространстве $H$.
%
%
%
Напомним следующие определения.
%
%
%
\begin{definition}
Билинейная форма $C$ называется \emph{\textbf{ограниченной}},
если найдется положительная постоянная $\hat{c}$
такая, что
для любых элементов $y, z \in H$ выполняется неравенство
\[
|C(y, z)|
\leq
\hat{c} \lVert y \rVert_H \lVert z \rVert_H.
\]
\end{definition}
%
%
%
\begin{definition}
Билинейная форма $C$ называется \emph{\textbf{коэрцитивной}},
если найдется положительная постоянная $\check{c}$
такая, что
для любого элемента $z \in H$ справедливо неравенство
\[
|C(z, z)|
\geq
\check{c} \lVert z \rVert_H^2.
\]
\end{definition}
%
%
%
\begin{proposition}
Билинейные формы $\mathcal{A}_u$, $\mathcal{B}_u$
являются ограниченными и коэрцитивными.
\end{proposition}
%
%
%
\begin{proof}
Напомним, что
\[
\rho \in L^\infty(I),
\qquad
u(x) \leq \beta.
\]
Тогда
\begin{equation}
\label{B:Boundedness}
|\mathcal{B}_u(y, z)|
=
\left|\int_0^1 \rho u y z \, dx
\right|
\leq
\int_0^1 \rho u |y z| \, dx
\leq
\beta
\lVert \rho \rVert_{L^\infty(I)}  \int_0^1 |y z| \, dx.
\end{equation}
Воспользовавшись известным неравенством Коши\nobreakdash-Буняковского
\cite{book:Kolmogorov,book:IP}
\[
\int_0^1 |y z| \, dx
\leq
\lVert y \rVert_{L^2(I)}
\lVert z \rVert_{L^2(I)},
\]
перепишем \eqref{B:Boundedness} в виде
\[
|\mathcal{B}_u(y, z)|
\leq
\beta
\lVert \rho \rVert_{L^\infty(I)}
\lVert y \rVert_{L^2(I)}
\lVert z \rVert_{L^2(I)}.
\]
Следовательно, билинейная форма $\mathcal{B}_u$ является ограниченной.
%
%
%
Аналогично получаем
\[
|\mathcal{A}_u(y, z)| = 
\left|
\int_0^1 eu^\nu y'' z'' \, dx
\right|
\leq
\int_0^1 eu^\nu |y'' z''| \, dx
\leq 
\beta^\nu
\lVert e \rVert_{L^\infty(I)}
\int_0^1 |y'' z''| \, dx.
\]
Снова применяя неравенство Коши\nobreakdash-Буняковского
теперь для выражения
\[
\int_0^1 |y'' z''| \, dx,
\]
заключаем,
что
\[
|\mathcal{A}_u(y, z)|
\leq
\beta^\nu
\lVert e \rVert_{L^\infty(I)}
\lVert y'' \rVert_{L^2(I)}
\lVert z'' \rVert_{L^2(I)}.
\]
Очевидно,
что
\[
\lVert y'' \rVert_{L^2(I)}
=
\sqrt{\int_0^1 |y''|^2 \, dx}
\leq
\left(\int_0^1 \left( |y|^2 + |y'|^2 + |y''|^2 \right) \, dx\right)^{1/2}
= \lVert y \rVert_{H^2(I)}.
\]
Таким образом,
\[
|\mathcal{A}_u(y, z)|
\leq
\beta^\nu
\lVert e \rVert_{L^\infty(I)}
\lVert y \rVert_{H^2(I)}
\lVert z \rVert_{H^2(I)},
\]
т.\,е.
билинейная форма
$\mathcal{A}_u$ является ограниченной.
%
%
%
Поскольку
\[
\rho(x) \geq \rho_0,
\qquad
u(x) \geq \alpha,
\]
то
\[
\mathcal{B}_u(z, z)
=
\int_0^1 \rho u z^2 \, dx
\geq
\alpha \rho_0
\int_0^1 z^2 \, dx
=
\alpha \rho_0 \lVert z \rVert_{L^2(I)}^2,
\]
что означает коэрцитивность билинейной формы $\mathcal{B}_u$,
так как
$\alpha, \rho_0 > 0$.
%
%
%
Коэрцитивность билинейной формы $\mathcal{A}_u$ для всех рассматриваемых реализаций пространства $V$
показана в
\cite{book:Fichera}.
\end{proof}
%
%
%
Ограниченность и коэрцитивность билинейных форм
$\mathcal{A}_u$ и $\mathcal{B}_u$ означает справедливость следующего известного утверждения.
%
%
%
\begin{theorem}
Задача
\eqref{BeamVibrating:WeakSetting}
имеет счетное множество собственных чисел
\[
0 < \lambda_1[u] \leq \lambda_2[u] \leq \ldots
\leq 
\lambda_k[u] \leq \ldots,
\qquad
\lim_{k \to \infty} \lambda_k[u] = +\infty
\]
и соответствующую последовательность собственных функций $\{ y_k \}$,
образующую ортогональный базис в $V$ и $L^2(I)$.
%
%
%
Кроме того, справедлив следующий вариационный принцип:
\begin{equation}
\label{FirstEigenvalue:VariationalPrinciple}
\lambda_1[u]
=
\min_{y \in V \setminus \{ \vartheta \}} \frac{\mathcal{A}_u(y,y)}{\mathcal{B}_u(y,y)}
=
\frac{\mathcal{A}_u(y_1,y_1)}{\mathcal{B}_u(y_1,y_1)}.
\end{equation}
%
%
%
Наконец, каждое собственное значение
$\lambda_k[u]$ имеет конечную кратность.
\end{theorem}
%
%
%
\noindent\emph{Доказательство} теоремы приводится в \cite{book:Litvinov}.
%
%
%
\par
Отметим,
что в литературе по механике (см., например,
\cite{book:Banichuk}) вариационное представление
\eqref{FirstEigenvalue:VariationalPrinciple}
первого собственного значения $\lambda_1[u]$
принято называть \emph{\textbf{вариационным принципом Рэлея}}.
%
%
%
Вместе с тем выражение
\[
\frac{\mathcal{A}_u(\cdot,\cdot)}{\mathcal{B}_u(\cdot,\cdot)}
\]
именуют \emph{\textbf{отношением Рэлея}}.
%
%
%
Вариационный принцип Рэлея означает,
что для любого ненулевого элемента $z$ из пространства состояний
$V$ выполняется неравенство
\[
\lambda_1[u]
=
\frac{\mathcal{A}_u(y_1,y_1)}{\mathcal{B}_u(y_1,y_1)}
\leq
\frac{\mathcal{A}_u(z,z)}{\mathcal{B}_u(z,z)},
\]
которое также можно записать в следующем виде:
\[
\lambda_1[u]
=
\frac
{\int_0^1 eu^\nu |y_1''|^2 \, dx}
{\int_0^1 \rho u y_1^2 \, dx}
\leq
\frac
{\int_0^1 eu^\nu |z''|^2 \, dx}
{\int_0^1 \rho u z^2 \, dx}.
\]
%
%
%
%
%
\par
Что касается кратности собственных значений 
задачи
\eqref{BeamVibrating:WeakSetting},
то
в действительности, как следует из
\cite{article:FourthOrderOscillations},
справедлива следующая
\begin{theorem}
Все собственные значения задачи
\eqref{BeamVibrating:WeakSetting}
являются простыми,
т.\,е.
\[
0 < \lambda_1[u] < \lambda_2[u] < \ldots
<
\lambda_k[u] < \ldots.
\]
\end{theorem}
%
%
%
В дальнейшем нам также понадобятся свойства собственных функций
задачи
\eqref{BeamVibrating:WeakSetting}
(см. \cite{article:FourthOrderOscillations}).
\begin{theorem}
Собственная функция,
соответствующая первому собственному значению задачи
\eqref{BeamVibrating:WeakSetting},
не меняет знака на интервале $I$.
\end{theorem}
%
% \section{Метод последовательных приближений определения собственной пары}
\label{section:ApproximationsMethod}


\begin{definition}
Первое собственное значение $\lambda_1[u]$ краевой задачи
\eqref{BeamVibrating:WeakSetting},
а также соответствующую ему собственную функцию $y$
будем называть \emph{\textbf{собственной парой}} этой задачи и записывать через $(\lambda_1[u], y)$.
\end{definition}
%
%
%
\par
В дальнейшем при численном решении задачи оптимального проектирования колеблющейся балки
(которая рассматривается в Главе~\ref{chapter:OptimalBeamDesign})
нам понадобится итерационный метод для приближенного определения собственной пары задачи
\eqref{BeamVibrating:WeakSetting}.
%
%
%
С этой целью будем применять стандартный \emph{\textbf{метод последовательных приближений}},
который состоит в построении последовательности
приближений $\{ (\lambda_i, y_i) \}$ собственной пары,
элементы которой определяются из следующих формул:
\[
(eu^\nu y_{i + i}'')'' = \rho u y_i,
\qquad
\lambda_i = \Lambda(u, y_i),
\]
где в качестве начального приближения $y_0$ берется дважды непрерывно дифференцируемая функция,
удовлетовряющая заданным краевым условиям,
а
\[
\Lambda(u, z)
\triangleq
\frac{\mathcal{A}_u(z,z)}{\mathcal{B}_u(z,z)}
=
\frac
{\int_0^1 eu^\nu |z''|^2 \, dx}
{\int_0^1 \rho u z^2 \, dx}.
\]
%
%
%
Сходимость метода последовательных приближений
обосновывается в
\cite{book:Collatz}.

%
% ПОСТАНОВКА ЗАДАЧИ ОПТИМАЛЬНОГО ПРОЕКТИРОВАНИЯ %%%
% \section{Постановка задачи оптимального проектирования}
\label{section:OptimalControlProblem}
%
%
%
%
%
\par
Условие постоянства массы запишем в виде
\begin{equation} 
\label{Condition:ConstantV}
m[u] = \int^1_0{\rho u(x)dx}.
\end{equation}
%
%
%
Тогда множество допустимых управлений имеет вид
\[
\mathcal{U}
=
\{
u \in U : \;
m[u] \leq 1
\}.
\]
Задача максимизации фундаментальной частоты свободных колебаний балки состоит в следующем:
найти допустимое распределение площади поперечных сечений $w \in \mathcal{U}$
такое, что
\[
\lambda_1[w] =
\sup_{u \in \mathcal{U}}
\lambda_1[u].
\]
%
%
%
Заметим,
что эта экстремальная задача представляет собой задачу оптимального управления коэффициентами
обыкновенного дифференциального уравнения,
в которой функционалом является первое собственное значение.
%
%
%
С другой стороны,
это задача о максимине, так как
\[
\sup_{u \in \mathcal{U}}
\lambda_1[u] =
\sup_{u \in \mathcal{U}}
\min_{y \in V \setminus \{ \vartheta \}} \frac{\mathcal{A}_u(y,y)}{\mathcal{B}_u(y,y)}
=
\sup_{u \in \mathcal{U}}
\min_{y \in V \setminus \{ \vartheta \}}
\frac
{\int_0^1 eu^\nu |y''|^2 \, dx}
{\int_0^1 \rho u y^2 \, dx}.
\]
%
%
%
Как известно,
задачи оптимального управления не всегда обладают решением,
так как
точная верхняя грань значений целевого функционала
может не достигаться.
%
%
%
Исследуемая в этой работе задаче
обладает решением.
Доказательство существования приводится в
\cite{article:Existence}.
%
% ИТЕРАЦИОННЫЙ МЕТОД РЕШЕНИЯ ЗАДАЧИ ОПТИМАЛЬНОГО ПРОЕКТИРОВАНИЯ %%%
% \section{Итерационный метод решения задачи оптимального проектирования}
\label{section:IterationMethod}

В
\cite{article:Problems}
установлено следующее утверждение.
\begin{theorem}
Оптимальное решение $w$ и собственная функция $y_1$, соответствующая собственному значению $\lambda_1[w]$,
образует седловую точку функционала $\Lambda(\cdot, \cdot)$,
т.\,е.
выполняются следующие неравенства:
\[
\Lambda(u, y_1)
\leq
\Lambda(w, y_1)
\leq
\Lambda(w, z),
\]
где
$z \in V$, $u \in \mathcal{U}$.
\end{theorem}
%
%
%
\par
В соответствии с приведенным утверждением оптимальное решение будем искать с помощью следующей итерационной
процедуры:
\[
\begin{gathered}
\Lambda(u_{n + 1}, y_n)
=
\sup \Lambda(u, y_n),
\\
\Lambda(u_n, y_n)
=
\min \Lambda(u_n, y).
\end{gathered}
\]
Сходимость этой процедуры обосновывается по схеме близкой к
\cite{article:Goncharov:Wing}.

