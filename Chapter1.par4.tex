\section{Постановка задачи оптимального проектирования}
\label{section:OptimalControlProblem}
%
%
%
%
%
\par
Условие постоянства массы запишем в виде
\begin{equation} 
\label{Condition:ConstantV}
m[u] = \int^1_0{\rho u(x)dx}.
\end{equation}
%
%
%
Тогда множество допустимых управлений имеет вид
\[
\mathcal{U}
=
\{
u \in U : \;
m[u] \leq 1
\}.
\]
Задача максимизации фундаментальной частоты свободных колебаний балки состоит в следующем:
найти допустимое распределение площади поперечных сечений $w \in \mathcal{U}$
такое, что
\[
\lambda_1[w] =
\sup_{u \in \mathcal{U}}
\lambda_1[u].
\]
%
%
%
Заметим,
что эта экстремальная задача представляет собой задачу оптимального управления коэффициентами
обыкновенного дифференциального уравнения,
в которой функционалом является первое собственное значение.
%
%
%
С другой стороны,
это задача о максимине, так как
\[
\sup_{u \in \mathcal{U}}
\lambda_1[u] =
\sup_{u \in \mathcal{U}}
\min_{y \in V \setminus \{ \vartheta \}} \frac{\mathcal{A}_u(y,y)}{\mathcal{B}_u(y,y)}
=
\sup_{u \in \mathcal{U}}
\min_{y \in V \setminus \{ \vartheta \}}
\frac
{\int_0^1 eu^\nu |y''|^2 \, dx}
{\int_0^1 \rho u y^2 \, dx}.
\]
%
%
%
Как известно,
задачи оптимального управления не всегда обладают решением,
так как
точная верхняя грань значений целевого функционала
может не достигаться.
%
%
%
Исследуемая в этой работе задаче
обладает решением.
Доказательство существования приводится в
\cite{article:Existence}.