\chapter*{Заключение}
\addcontentsline{toc}{chapter}{Заключение}
\thispagestyle{fancy}
\label{chapter:Conclusion}



































% \begin{enumerate}
% 	\item
% 	Исследована задача максимизации фундаментальной частоты свободных колебаний балки, решение которой позволяет находить оптимальные формы балки, при которых она наименее подвержена резонансу.
%
%
%
% 	Поэтому результаты данной работы могут найти практическое применение.
%
%
%
% 	\item
% 	В работе сформулирована итерационная процедура поиска седловой точки, позволяющая решить задачу максимизации фундаментальной частоты свободных колебаний балки.
%
%
%
% 	Этот подход к решению данной задачи ранее в литературе не применялся и поэтому является новым.
%
%
%
% 	Поэтому можно ожидать, что предложенный метод позволяет строить сходящиеся к оптимальному решению последовательности.
%
%
%
% 	\item
% 	Для этой итерационной процедуры был реализован численный метод в виде библиотеки на языке програмирования Python, а так же проведены вычислительные эксперементы, в ходе которых последовательность приближений демонстрировала сходимость.
%
%
%
% \end{enumerate}
%
%
