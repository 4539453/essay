\documentclass[oneside,a4paper,14pt,final]{extreport}

\usepackage{essay}
%%%%%%%%%%%%%%%%%%%%%%%%%%%%%%%%%%%%%%%%%%%%%%%%%%%
%                                                 %
%              declare math operators             %
%                                                 %
%%%%%%%%%%%%%%%%%%%%%%%%%%%%%%%%%%%%%%%%%%%%%%%%%%%
% https://tex.stackexchange.com/questions/67506/newcommand-vs-declaremathoperator
% \DeclareMathOperator{\End}{End}
\DeclareMathOperator*{\argmax}{arg\,max}
\DeclareMathOperator*{\argmin}{arg\,min}
\DeclareMathOperator*{\lh}{Likelihood}
\DeclareMathOperator*{\sign}{sign}

\graphicspath{ {./graphics/} }
\addbibresource{./chapters/library.bib}

\begin{document}
%%%%%%%%%%%%%%%%%%%%%%%%%%%%%%%%%%%%%%%%%%%%%%%%%%%
\chapter*{РЕФЕРАТ}

% Использование нейронных сетей и методов машинного обучения для сентимент-анализа текстов

% Введение
%   обосновывается актуальность ВКР и практическая значимость
%   объект, предмет, цель и задачи ВКР
%   определяются методы исследования
%   краткий обзор информационной базы исследования.

% Теоретическая часть
%   Обзор предметной области (включая краткий обзор этих исследования)
%       По количеству классов
%       По подходам к анализу
%   Задача классификации и сентимент-анализа текстов

%   Подготовка данных?

%   Обработка текстов
%       Основная идея архитектуры нейронных сетей
%           Функции активации
%       Дистрибутивная семантика
%       Распределенные представления слов word2vec
%       ELMo

%   Классификация
%       Macro F1
%       Cross validation
%       log reg
%       SVM
%       random forest

% Практическая часть
%   Используемые иструменты
%   Сбор данных
%   Модели классификации
%       1
%       2
%   Эксперименты

% Заключение
%   общие результаты ВКР
%   обобщённые выводы и предложения, указываются
%    перспективы применения результатов на практике
%    возможности дальнейшего исследования проблемы

%
% Список источников
%
% Приложение
%   Toloka
%   Models

\par
Выпускная квалификационная работа содержит \pageref*{LastPage}~страниц, \totfig~рисунка,                                        \tottab~таблицу, XX использованных источника.
\bigskip

\par
КЛЮЧЕВОЕ СЛОВО 1, КЛЮЧЕВОЕ СЛОВО 2, …
\bigskip


\par
Краткое описание содержания работы.


\thispagestyle{empty} % unnumbered page

{\linespread{0.9}\tableofcontents}

\chapter*{Введение}
\addcontentsline{toc}{chapter}{Введение}

\par
Классификация текстов --- невероятно популярная задача. Мы пользуемся текстовыми классификаторами в почтовом клиенте: он классифицирует письма и фильтрует спам. Другие приложения включают классификацию документов, обзоров и так далее.

\bigskip\par
Обычно классификация текстов используется не как самостоятельная задача, а является частью более крупного пайплайна. Например, голосовой помощник классифицирует ваше высказывание, чтобы понять, что вы хотите (например, установить будильник, заказать такси или просто поболтать) и передать ваше сообщение другой модели в зависимости от решения классификатора.  Другой пример --- поисковый движок: он может использовать классификаторы для определения языка запроса, чтобы предсказать его тип (например, информационный, навигационный, транзакционный) и понять хотите ли вы увидеть картинки или видео помимо документов и прочего.

\bigskip\par
Моя работа посвящена одному из приложений классификации --- автоматическому определению эмоциональной оценки русскоязычных текстов. Главная особенность  заключается в том, что предсказание базируется на эмоциональных моделях, предложенных Робертом Плутчиком и Полом Экманом.


\bigskip\par
Поскольку большинство датасетов для классификации содержат только одну правильную метку. А у нас многоклассовая классификация.

%%%%%%%%%%%%%%%%%%%%%%%%%%%%%%%%%%%%%%%%%%%%%%%%%%%
{\CenterChapterHeading\chapter*{ОСНОВНАЯ ЧАСТЬ}
\addcontentsline{toc}{chapter}{ОСНОВНАЯ ЧАСТЬ}
\newpage
}
%%%%%%%%%%%%%%%%%%%%%%%%%%%%%%%%%%%%%%%%%%%%%%%%%%%
\section{ТЕОРЕТИЧЕСКАЯ ЧАСТЬ}

\textcolor{red}{В этой части мы подробнее раскроем теорию.}

\subsection{Обзор предметной области}


\par
Классификация текстов --- невероятно популярная задача. Мы пользуемся текстовыми классификаторами в почтовом клиенте: он классифицирует письма и фильтрует спам. Другие приложения включают классификацию документов, обзоров и так далее.

\bigskip\par
Обычно классификация текстов используется не как самостоятельная задача, а является частью более крупного пайплайна. Например, голосовой помощник классифицирует ваше высказывание, чтобы понять, что вы хотите или чувствуете и передать ваше сообщение другой модели в зависимости от решения классификатора.  Другой пример --- поисковый движок: он может использовать классификаторы для определения языка запроса, или предсказать его тип (например, развлекательный, информационный, навигационный, транзакционный) и понять хотите ли вы увидеть картинки или видео помимо документов или подобрать контент по настроению.

% \bigskip\par
% Поскольку большинство датасетов для классификации содержат только одну правильную метку. А у нас многоклассовая классификация.
\bigskip
Задачи интеллектуальной обработки текста делятся на три типа: синтаксические, основанные на понимании смысла и третий --- не просто понимание написанного, а еще и генерация нового текста. Сентимент-анализ относится ко второму типу.
%%% 1. Semina %%%
\bigskip\par
Моя работа посвящена одному из приложений классификации --- автоматическому определению эмоциональной оценки русскоязычных текстов. Главная особенность  заключается в том, что предсказание базируется на эмоциональных моделях, предложенных Робертом Плутчиком и Полом Экманом.

\bigskip\par
Сентимент-анализ, как направление компьютерной лингвистики, стал очень популярен последние десятилетие. С
появлением больших данных насыщенных эмоциональной составляющей возникла потребность в их обработке. Компании
начали устраивать соревнования с внушительными призовыми фондами, а исследователи искать лучшую архитектуру
для классификации этих данных. Так в открытом доступе появились большие размеченные датасететы с отзывами,
данными из соцсетей и новостями.

\bigskip\par
Сам термин <<сентимент>> означает совокупность чувств и взглядов, как основа для действия или суждения; общая
эмоциональная установка. Целью сентимент-анализа является выделении этих тональных компонент из текста.
Рассмотрим его применение на разных уровнях \cite{Semina}.

\bigskip\par
Пусть есть целый документ, тогда, как правило, в нем можно выделить один субъект и один объект, а так же
сентимент. Ярким примером такого уровня является отзыв. Здесь автор выступает в качестве субъекта, а предмет
отзыва --- в качестве объекта. Это уровень документа.

\bigskip\par
Если документ более сложный, то можно рассматривать его на уровне предложений. В результате можно определить
эмоциональную окраску всего документа или предложений по отдельности. Зависит от поставленной задачи.

\bigskip\par
Также анализ бывает на уровне аспектов. Смысл его в том, что эмоциональная установка определяется не для
конкретно объекта, а для его отдельных составляющих --- аспектов. Например, для объекта <<компьютер>> можно
выделить аспекты <<производительность>>, <<дизайн>>, <<сборка>> т.д., другими словами, к аспектам относится
все то, к чему могут быть выражены эмоции. Данная задача очень востребована, потому что позволяет точнее
определять отношение автора к объекту, а для некоторых задач это очень важный критерий.

\bigskip\par
Последний вид анализа самый сложный и проводится уровне именованных сущностей (Named Entities). Именованная
сущность --- это абстрактный или физический объект, который может быть обозначен собственным именем. Сама по
себе задача извлечения именованных сущностей (Named Entity Recognition) не из простых, а вкупе с
сентимент-анализом становится действительно комплексным решением.

% https://habr.com/ru/company/mailru/blog/516214/
% https://ieeexplore.ieee.org/document/9117010/references#references

\bigskip\par
Методы сентимент анализа можно также разделить на несколько основных направлений \cite{Semina}:
\bigskip
\begin{itemize}
 \item \textit{метод основанный на правилах (rule-based).} Здесь используются наборы правил классификаци, составленные экспертом и эмоционально размеченные словари. Определенный класс присваивается в зависимости от найденных ключевых слов и их использования с другими ключевыми словами. Такой метод достаточно эффективный, но
 очень трудоемкий. Неплохих результатов в бинарной классификации добились в работе \cite{atex};

 \item \textit{метод основанный на словарях.} Самый простой метод, основанный  на подсчете сентиментных единиц содержащихся в словарях тональности. Очень сильно зависит от размера словаря и не очень точен в разрыве с правилами русского языка. Попытка получения сентимента из текста таким способом описана в этой работe \cite{Kirill};

 \item \textit{методы основанные на машинном и глубоком обучении.} (machine learning, deep learning) Наиболее популярная группа методов в сентимент-анализе, работающих на способности алгоритмов обобщать выделенные из текста признаки. Их применение позволило получить очень высокие результаты в работах \cite{senti1, senti2, senti3, senti4};

 \item \textit{гибридные методы.} Позволяют одновременно использовать несколько подходов. % ++ дописать сюжа
что-нибудь
\end{itemize}

\bigskip\par
Разберем подробнее машинное и глубокое обучение. Суть метода в выделении признаков из текста и последующем их
обобщении с помощью разнообразных алгоритмов. Для выделения признаков используют, как простые алготитмы по
типу мешок слов (Bag of Words) или TF-IDF, так и небольшие нейронные сети для генерирования эмбеддингов,
например, Word2Vec \cite{Mikolov:1}, GloVe \cite{glove}, FastText \cite{fasttext}. Существуют и более сложные алгоритмы, которые формируют признаки на уровне предложений, к ним относятся ELMo \cite{elmo1,elmo2}, BERT (Bidirectional Encoder Representations from Transformers) \cite{bert} и др.

\bigskip\par
Чтобы обработать выделенные признаки используют разнообразные алгоритмы. К классическому обучению относятся:

\bigskip
\begin{itemize}
 \item байессовский классификатор (Naive Bayes classifier);
 \item дерево решений (Decision Tree);
 \item логистическая регрессия (Logistic Regression);
 \item метод опорных векторов (Support Vector Machine).
\end{itemize}

\bigskip\par
Среди алгоритмов глубокого обучения можно выделить:
\begin{itemize}
 \item реккурентные нейронные сети (RNN);
 \item сверточные нейронные сети (CNN);
 \item полносвязные нейронные сети (FCNN) и т.д.
\end{itemize}


\subsection{Задача классификации и сентимент-анализа текстов}

Пусть есть описание документа $d \in X$, где $X$ --- векторное пространство документов и фиксированный набор меток $C = \{c_1, c_2, \ldots, c_n\}$. Из обучающей выборки (множества документов с заранее известными метками --- эмоциями) $D = \{\langle d, c \rangle | \langle d, c \rangle\ \in X \times C\}$ и с помощью метода обучения $G$ необходимо получить классифицирующую функцию $G(D) = f$, которая отображает документы в пространство меток $f : X \to C$.


\subsection{Данные и предобработка текстов}







\subsection{Основная идея архитектуры нейронных сетей}

Чтобы объяснить работу неронных сетей нужно опредить из чего они состоят. Для этого рассмотрим простейший
линейный бинарный классификатор -- перцептрон Розенблата. Пространство данных разделяется на два множества
гиперплоскостью, а метка класса будет ставиться в зависимости от значения линейной функции от входных
признаков.

\begin{equation}
 \sign (w_0 + w_1x_1 + w_2x_2 + \ldots + w_dx_d),
\end{equation}

где $x =(x_1, x_2, \ldots, x_d) \in \mathds{R}^d$. Мы ищем такие веса $w_0, w_1 \ldots, w_d \in \mathds{R}^d$,
чтобы $\sign$ от скалярного произведения признаков и весов $w^\top x$ совпадал с верной меткой $y(x) \in
{-1,1}$, но для этого добавим фиктивную переменную в вектор $x =(1, x_1, x_2, \ldots, x_d) \in
\mathds{R}^{d+1}$, чтобы размерности сохранялись.

\bigskip
Теперь обучим эту функцию, для этого нам нужна функция ошибки, она называется критерий Перцептрона:

\begin{equation}
 L_P(w) = - \sum_{x \in M} y(x)(w^\top x),
\end{equation}

где $M$ --- множество неверно классифицируемых примеров. В качестве оптимизатора выберем градиентный спуск. С
помощью него мы минимизируем суммарное отклонение предсказаний классификатора от верных, но только в
неправильную сторону. Верное предсказание никак не влияет на функцию ошибки. В результате получается
кусочно-линейная функция, которая почти везде дифференцируема и этого достаточно для применения градиентного
спуска. Процесс обучение выглядит так: если предсказание верное, то не делаем ничего, если классификатор
ошибся, то делаем градиентный шаг.

Такая модель перцептрона линейная и результат ее работы не слишком содержателен. Чтобы из перцепртонов можно
было составить сеть, нужно добавить нелинейность. Такой нелинейностью будет функция активации. Они бывают
разные, самая распространенная --- сигмоида рис. \ref{fig:sigmoid}:

\begin{equation} \label{eq:sigma}
    \sigma (x) = \frac{1}{1+e^{-x}}
\end{equation}

\begin{figure}[ht]
    \centering
    \includegraphics[scale=0.8]{sigmoid.png}
    \caption{Сигмоида}
    \label{fig:sigmoid}
\end{figure}

Обучить этот прецептрон также не составляет труда. Просто теперь мы будем решать задачу бинарной
классификации, а функция ошибки будет cross-энтропией:

\begin{equation}
 L(w) = -\frac{1}{N}\sum_{i=1}^N(y_i\log\sigma(w^\top x_i) + (1-y_i)\log(1-\sigma(w^\top x_i))).
\end{equation}

Эта функция дифференцируема, значит мы можем сделать градиентный шаг. При этом прецептрон с сигмоидой
реализует логистическую регрессию. Обобщение этой модели можно найти в разделе \ref{subsection:logreg}.

Графическое изображение структуры перцепрона представлено на рис. \ref{fig:neuron}, \textit{a}.

\begin{figure}[ht]
    \centering
    \includegraphics[scale=0.8]{neuron.png}
    \caption{(a) граф вычислений перцептрона; (б) полносвязная нейронная сеть с одним скрытым слоем.}
    \label{fig:neuron}
\end{figure}

На  рис. \ref{fig:neuron}, \textit{б} изображена несложная нейронная сеть, на ее примере покажем как можно
векторизовать вычисления в слое нейронов с применением функции активации.

Пусть у нас в слое $k$ нейронов с весами $w_1, w_2, \ldots, w_k$, $\quad w_i = (w_{i1}, \ldots, w_{in})^\top$,
на вход вектор $x = (x_1, x_2, \ldots, x_n)^\top$. В результате получим выход $y_i = f(w_i^\top x)$, где $f$
--- функция активации. Эти вычисления можно представить в векторной форме:

\begin{equation*}
\begin{pmatrix}
   y_1 \\
   \ldots \\
   y_k
\end{pmatrix} = y = f(Wx) =
\begin{pmatrix}
   f(w_1^\top x)\\
   \ldots \\
   f(w_k^\top x)
\end{pmatrix}\text{, где } W =
\begin{pmatrix}
   w_1 \\
   \ldots \\
   w_k
\end{pmatrix} =
\begin{pmatrix}
   w_{11} & \ldots & w_{1n}\\
   \ldots &  & \ldots\\
   w_{kk} & \ldots & w_{kn}\\
\end{pmatrix}.
\end{equation*}


\subsubsection{Функции активации}

Обычно в нелилинейных неронах примемяется сигмоида \ref{eq:sigma}, о которой писалось выше. Но существуют и
другие функции активации, например, $\text{ReLU} = max(0, x)$. Чтобы понять, как она появилась рассмотрим
сумму бесконечного ряда сигмоид, каждая из которых смещена на единицу относительно предыдущей:

\begin{equation*}
 f(x) = \sigma(x+\frac{1}{2} + \sigma(x-\frac{1}{2} + \sigma(x-\frac{3}{2}) + \ldots
\end{equation*}

$f(x)$ можно представить в виде интегала:

\begin{equation*}
 \int_\frac{1}{2}^\infty \sigma(x+\frac{1}{2}-y)dy,
\end{equation*}

Его значение можно приюблизить единичными прямоугольниками:

\begin{equation*}
\begin{aligned}
  \sum_{i=0}^\infty \sigma(x+\frac{1}{2}-i) & \approx \int_\frac{1}{2}^\infty \sigma(x+\frac{1}{2}-y)dy = \\
  & = \left[-\log (1+e^{x+\frac{1}{2}-y})\right]^{y=\infty}_{y=\frac{1}{2}} = \log (1+e^x),
\end{aligned}
\end{equation*}

а это в точности интеграл от сигмиоды:

\begin{equation*}
 \int \sigma (x) dx = \log(1+e^x) + C.
\end{equation*}

Получается бесконечный ряд сигмоид гораздо более выразительная функция активации и это почти тоже самое что и
$\log(1+e^x)$



% Это приближение к ф-ии log(1+e^x) (SoftPlus), которая получается из суммы бесконечного ряда сигмоид со
смещением на 1/2 и тд. Это дает возможность отличать сильно активированные нейроны от слабо активированных.
Но считать производную такой функции очень накладно, поэтому ее приближение -- хороший компромисс. Существуют
еще разные модификации ReLU, например, PReLU или LReLU. В целом она показала лучшую эффективность по сравнению
с сигмоидой и tanh, следовательно используется повсеместно.
%
% SoftMax или нормализованная экспонента. Она нужна чтобы обобщить функцию ошибки для задач классификации,
где классов больше 2-х. Ее удобно использовать на последних слоях нейронной сети, чтобы преобразовать выход в
<<вероятность>>.


































\subsection{Обработка текстов}
\label{section:textRepsesentation}

То, как модели видят данные отличается от того, как из видят люди. Например, мы легко можем понять
предложение <<Да будет свет>>, но модели не могут --- им нужны векторы с признаками. Такие векторы являются
представлениями слов, которые может обработать наша модель.

\bigskip
Самой простой формой представления слов является дискретное представление, т.е. one-hot представление: все
слова представляются в виде вектора, размерность которого совпадает с числом слов словаре. Причем все
компоненты кроме $i$-го равны нулю, а позиция, соответствующая $i$-му слову равна единице. Очевидно, что такой
способ не самый лучший. Во-первых такое представление зависит от положения слов в словаре, а это нежелательно,
потому что задает бессмысленные отношения между словами. Во-вторых размеры такого словаря растут прямо
пропорционально количеству слов в нем, а это значит размерность его может доходить до сотен тысяч и работать с
ними станет очень вычислительно накладно. В-третьих такое представление совершенно не учитывает значение слов,
для решения этой проблемы обратимся к дистрибутивной семантике.

\subsubsection{Дистрибутивная семантика}

Чтобы зафиксировать значение слов в их векторах, нам нужно сначала определить понятие значения. Для этого
возмем несколько предложений

\bigskip
\begin{itemize}
 \item возтмем несколько предожений
 \item построим вектора
 \item способы из сравнить (косинусное расстояние)
\end{itemize}



\subsubsection{Методы основанные на подсчетах}


\subsubsection{Распределенные представления слов}

Подход к обучению моделей распределенных представлений слов был описан в работе Йошуа Бенджи с соавторами
\cite{Bengio}, которая была продолжена в \cite{Zhou}. Идея подхода описанного в \cite{Bengio} основанна на
задаче построения языковой модели, процесс обучения выглядит так:

\bigskip
\begin{itemize}
 \item всем токенам из словаря $i \in V$ ставят в соответствие вектор признаков $w_i$ размерности $d$ ($w_i
\in \mathds{R}^d$). Стандартным значением $d$ является 300;

 \item теперь можно определить вероятности для каждого токена $i$, что он появится в контексте $c_1, \ldots,
c_n$. Для этого определим функцию от векторов признаков $w$:

 \begin{equation}
  \hat{p}(i|c_1, \ldots, c_n) = f(w_i, w_{c_1}, \ldots, w_{c_n}; \theta),
 \end{equation}

 где $w_{c_1}, \ldots, w_{c_n}$ --- векторы признаков токенов из контекста, a $f$ --- фунция с параметрами
$\theta$, которая принимает векторы признаков;

 \item максимизируя логарифм правдоподобия большого корпуса текстов можно обучить векторы признаков $w$ и
параметры $\theta$

 \begin{equation}
  L(W, \theta) = \frac{1}{K}\sum_t \log{f(w_k, w_{k-1}, \ldots, w_{k-n+1}; \theta) + R(W, \theta)},
 \end{equation}

 где $K$ размер окна контекста, а $R(W, \theta)$ --- регуляризация.

\end{itemize}

\bigskip
Для получения функции $f$ можно использовать нейронную сеть. Модель word2vec строится на описанииf
нейросетевой модели, предложенной в \cite{Bengio}. Она была разработана Томасом Миколовым с соавторами и
опубликована в работах \cite{Mikolov:1, Mikolov:2}, причем в двух вариациях:

\bigskip
\begin{itemize}
 \item CBOW (Continious Bag Of Words) --- по контексту восстановить слово;
 \item skip-gram --- восстановить контекст в зависимости от слова;
\end{itemize}

\bigskip
Архитектура word2vec представляет собой полносвязную нейронную сеть с одним скрытым слоем рис.
\ref{fig:word2vec}.

\begin{figure}[ht]
    \centering
    \includegraphics[scale=0.8]{word2vec.png}
    \caption{(a) CBOW; (б) skip-gram.}
    \label{fig:word2vec}
\end{figure}

% !! смещение
Принцип работы модели CBOW рис. \ref{fig:word2vec}, \textit{а}  выглядит так:

\bigskip
\begin{itemize}
 \item на вход сети подаются one-hot вектора размерности $V$, где V --- это размер словаря;
 \item скрытый слой --- это матрица $W$ размерности $V\times d$, которая переводит наши представления слов в
$d$ - мерное пространство;
 \item на выходе для каждого слова в словаре берем среднее всех полученных векторов и получаем оценку $u_j$,
где $j = 1, \ldots, V$.
\end{itemize}

\bigskip
Чтобы найти апострериорное распределение модели, просто вычисляем softmax:

\begin{equation}
 \hat{p}(i|c_1, \ldots, c_n) = \frac{\exp{u_j}}{\sum_{j'=1}^V \exp{u_{j'}}}.
\end{equation}

Для аппроксимации апостериорным распределением распределения данных используем loss-функцию для одного окна:

\begin{equation}
 L = -\log{p(i|c_1, \ldots, c_n)} = - u_j + \log{\sum_{j'=1}^{V} \exp{u_{j'}}}.
\end{equation}

Принцип работы модели skip-gram рис. \ref{fig:word2vec}, \textit{б} полностью противоположный. До этого мы
усредняли контекст, чтобы получить среднее слово в окне, а теперь будем предсказывать слова контекста исходя
из центрального слова. На выходе мы получаем $K-1$ мультиномиальных распределений (центральное слово не
учитывается):

\begin{equation}
 \hat{p}(c_k|i) = \frac{\exp{u_{kc_k}}}{\sum_{j'=1}^V \exp{u_{j'}}},
\end{equation}

loss-функция для окна размера K выглядит так:

\begin{equation}
 L = -\log{p(c_1, \ldots, c_n|i)} = - \sum_{k=1}^{K} u_{kc_k} + n\log{\sum_{j'=1}^{V} \exp{u_{j'}}}.
\end{equation}

Возникает вопрос, как же обучить такую модель? Этот процесс хорошо описан в докладе Голдберга и его соавторов
\cite{Goldberg}.

\bigskip
Подробно разберем модель skip-gram для корпуса документов $D$. Нашей задачей стоит нахождение оптимальных
параметров модели $\theta$, чтобы максимизировать функцию правдоподобия:

\begin{equation} \label{eq:likelyhood}
 L(\theta)=\prod_{i\in D}\left(\prod_{c\in C(i)} p(c|i;\theta)\right) = \prod_{(i,c)\in D}p(c|i;\theta),
\end{equation}

где $C(i)$ --- множество контекстных слов внутри окна вокруг центрального слова $i$. Вероятность $p(c|i;\theta
)$ определяется, как softmax-функция, зависящая от всех возможных векторов контекста.

\begin{equation} \label{eq:generalSoftmax}
 p(c|i;\theta) = \frac{\exp{\tilde{w}_c^\top w_i}}{\sum_{c'} \exp{\tilde{w}_{c'}^\top w_i}},
\end{equation}

где $\tilde{w}_c$ --- вектор признаков слова из контекста $c$, который отличается от $w_i$. Для каждого слова
$i$ надо обучить два вектора признаков $w_i$ и $\tilde{w}_i$, в первом случае это слова выступает в качестве
центрального, во втором в качестве контекстного.

\bigskip
Эта особенность обучения, когда мы берем два разных вектора одного и того же слова вместо одного, описана в
\cite{Goldberg}. И мотивирована тем, что слова редко встречаются в контексте себя самих. Вот, например, слово
<<мотивация>> вряд ли можно встретить в контексте другого слова <<мотивация>>, под это правило попадают почти
все слова. Поэтому в процессе обучения модель сведет вероятности $p(i|i, \theta)$ к нулю. А если вектора
контекста и центрального слова будут равны нулю, то норма вектора $|w_i| = w_i^\top w_i$ тоже будет равняться
нулю, а это очень не желательно. Поэтому для каждого слова мы обучаем два разных вектора.

\bigskip
Теперь выразим максимум функции правдоподобия для всего корпуса через логарифм \ref{eq:likelyhood} и
\ref{eq:generalSoftmax}:

\begin{equation}
\begin{aligned}
 \argmax_{\theta} \prod_{(i,c)\in D} & p(c|i;\theta) = \argmax_{\theta} \sum_{(i,c)\in D}\log{p(c|i;\theta)} =
\\
 & = \argmax_{\theta} \sum_{(i,c)\in D} \left(\exp{\tilde{w}_c^\top w_i} - \log{\sum_{c'}
\exp{\tilde{w}_{c'}^\top w_i}}\right).
\end{aligned}
\end{equation}

Оптимизируя данную функцию мы получаем хорошее распределенное представление слов. Но для этого нужно решить
сложнейшую задачу: суммировать скалярные произведения всех возможных слов и их контекста $\sum_{c'}
\tilde{w}_c^\top w_i$ при том, что размер словаря может достигать миллионов.

\bigskip
Чтобы уменьшить количество вычислений Миколов с соавторами \cite{Mikolov:2} предложили элегантный метод:
negative sampling. Нам не нужно считать всю сумму $\sum_{c'} \tilde{w}_c^\top w_i$, а только случайно выбрать
несколько ее элементов в качестве отрицательных примеров (примеры в которых слово не находится в определенном
контексте) и обновить только их. Т.е. теперь нам нужно посчитать только небольшую сумму $\sum_{c' \in D'}
\tilde{w}_c^\top w_i$, где $D'$ --- случайное подмножество отрицательных примеров.

\bigskip
По сути negative sampling --- это тоже правдоподобие, но другого события. Пусть у нас есть слово $i$ и его
контекст $c$, наша задача максимизировать вероятность $p((i,c) \in D; \theta)$, параметризированную вектором
$\theta$, т.е. правдоподобие появления пары $(i,c)$:

\begin{equation}
 \argmax_{\theta} \prod_{(i,c)\in D} p((i,c)\in D;\theta) = \argmax_{\theta} \sum_{(i,c)\in D} \log{
p((i,c)\in D;\theta)}.
\end{equation}

Выразим $p((i,c)\in D;\theta)$ через softmax. Но так как это бинарное событие, то заменим softmax сигмоидой
$\sigma (x) = \frac{1}{1+\exp{(-x)}}$:

\begin{equation}
 p((i,c)\in D;\theta) = \frac{1}{1+\exp{(-\tilde{w}_c^{\top} w_i)}}
\end{equation}

Максимизируем логарифм правдоподобия:

\begin{equation} \label{eq:negLikelyhood}
\begin{aligned}
 \argmax_{\theta} \sum_{(i,c)\in D} & \log{p((i,c)\in D;\theta)} = \\
 & = \argmax_{\theta} \sum_{(i,c)\in D} \log{\frac{1}{1+\exp{(-\tilde{w}_c^{\top} w_i)}}}.
\end{aligned}
\end{equation}

Из \ref{eq:negLikelyhood} видно, что оптимальное значение логарифма будет получено при максимальном значении
скалярного произведения $\tilde{w}_c^{\top} w_i$. Сделаем равные векторы с большой нормой и можно без проблем
получить правдоподобие почти равное единице. Подвох заключается в том, что модель обучается на данных для
бинарной классификации, но мы рассматриваем только набор состоящий из положительных примеров. Классификатор,
который всегда предсказывает <<да>> --- плохой. Поэтому имеет смысл добавить отрицательных примеров, просто
случайно выбирая слова и контекст, которых нет в данных. После того, как мы получим набор отрицательных
данных, максимизация правдоподобия будет выглядеть так:

\begin{equation} \label{eq:generalLikelyhood}
 \argmax_{\theta} \prod_{(i,c)\in D} p((i,c)\in D;\theta) \prod_{(i,c)\in D'} p((i',c')\notin D;\theta)
\end{equation}

Выразим в \ref{eq:generalLikelyhood} пару $(i,c) \in D$:

\begin{equation}
\begin{aligned}
 & \argmax_{\theta} \prod_{(i,c)\in D} p((i,c)\in D;\theta) \prod_{(i,c)\in D'} 1-p((i',c')\in D;\theta) = \\
 = & \argmax_{\theta} \left[\sum_{(i,c)\in D} \log{p((i,c)\in D;\theta)} + \sum_{(i,c)\in D'} \log{(
1-p((i',c')\in D;\theta))} \right] = \\
 = & \argmax_{\theta} \sum_{(i,c)\in D} \left[\log{\frac{1}{1+\exp{(-\tilde{w}_c^{\top} w_i)}}} +
\sum_{(i,c')\in D'} \log{\frac{1}{1+\exp{(\tilde{w}_{c'}^{\top} w_i)}}} \right] = \\
 = & \argmax_{\theta} \sum_{(i,c)\in D} \left[\log\sigma(\tilde{w}_c^{\top} w_i) + \sum_{(i,c')\in D'}
\log\sigma(-\tilde{w}_{c'}^{\top} w_i)\right]
\end{aligned}
\end{equation}

Получили формулу для negative sampling из \cite{Mikolov:2}. Значит мы для каждого окна случайно берем
несколько отрицательных примеров $D'$ и делаем градиентный шаг для loss-функции:

\begin{equation}
 L = \log\sigma(\tilde{w}_c^{\top} w_i) \sum_{(i,c')\in D'} \log\sigma(-\tilde{w}_{c'}^{\top} w_i)
\end{equation}

Аналогичные рассуждения можно провести для модели CBOW.

% ++ можно добавить матричный подход

\subsubsection{GloVe}

Модель GloVe \cite{Pennington} представляет собой комбинацию методов на основе подсчета и методов предсказания (например, word2vec). Название модели GloVe расшифровывается как <<Global Vectors>>, что отражает ее идею: использование информации из всего корпуса для обучения векторов.








\section{Алгоритмы классического машинного обучения}

\subsection{Логистическая регрессия}
\label{subsection:logreg}

Логистическая регрессия --- классическая дискриминативной линейная модель классификации. Дискриминативная значит, что нас интересует $P (y = k | x)$, а не совместное распределение $p (x, y)$. Свое начало она берет из расстояния Кульбака-Лейблера. Оно задается формулой:

\begin{equation}
 KL(P||Q) = \int\log\frac{dP}{dQ}dP,
\end{equation}

, где $P$ --- истинное распределение, а $Q$ --- приближенное. Для дискретного случая:

\begin{equation}
 KL(P||Q) = \sum_{y} p(y)log\frac{p(y)}{q(y)},
\end{equation}

а если раскрыть получаем:

\begin{equation}
\begin{aligned}
 KL(P||Q) & = \sum_y p(y)\log\frac{p(y)}{q(y)} = \\
 & = \sum_y p(y) \log p(y) - \sum_y p(y) \log q(y) = - H(p) + H(p,q),
\end{aligned}
\end{equation}

, где $H(p)$ ---  энтропия распределения $p$, a $H(p, q)$ --- наша кросс энтропия. Из этой суммы видно, что нам нужно минимизировать $H(p, q)$. Для бинарной классификации loss-функция будет выглядеть так:

\begin{equation} \label{eq:logLoss}
 L(w)= H(p_{data}, q(w)) = -\frac{1}{N}\sum_{i=1}^N(y_i\log\hat y_i(w) + (1-y_i)\log(1-\hat y_i(w)))
\end{equation}

где $p_{data}$ --- распределение наших данных, $q(w)$ --- апостериорное распределение,  $\hat y_i(w)$ --- оценка вероятности при входных параметрах $w$ и $y_i$ --- истинное предсказание. Два слагаемых мы получаем, т.к. события несовместные. Например, в тексте говорится о кошечках или о собачках, события появления кошечки или собачки несовместные, т.е. $p(\text{кошечки}) = 1 - p(\text{собачки})$. Если мы предсказываем кошечку (1), как абсолютный 0 или собачку (0), как 1, то ошибка будет бесконечной из первого и второго слагаемых соответственно -- это не допустимо.

Перед тем, как перейти к нескольким классам, рассмотрим сначала задачу классификации с Байейсовской точки зрения: определим для каждого класса $C_k$ плотность $p(x|C_k)$ и какие-то априорные распределения $p(C_k)$ (пускай это будут размеры классов, т.е. мы ничего не знаем о примере, но предполагаем с какой вероятностью он относится к конкретному классу) и найдем $p(C_k|x)$. Для двух классов:

\begin{equation}
\begin{aligned}
 p(C_1|x) & = \frac{p(x|C_1)p(C_1)}{p(x|C_1)p(C_1)+p(x|C_2)p(C_2)} = \\
 & = 1/\frac{(p(x|C_1)p(C_1)+p(x|C_2)p(C_2))}{p(x|C_1)p(C_1)} = \\
 & = 1/(1+\frac{p(x|C_2)p(C_2)}{p(x|C_1)p(C_1)}) = \frac{1}{1+e^{-a}} = \sigma (a),
\end{aligned}
\end{equation}

где

\begin{equation}
 a = \ln\frac{p(x|C_1)p(C_1)}{p(x|C_2)p(C_2)},\qquad \frac{1}{1+e^{-a}} = \sigma (a).
\end{equation}

Используя логистическую регрессию мы делаем предположение о виде аргумента сигдмоиды $a$ --- это будет скалярное произведение вектора признаков на вектор данных: $a = w_\top x$. Cигмоида переводит результат вычисления этой линейной функции на отрезок $[0;1]$ и как результат мы получаем апостериорную вероятность первого или второго классов:

\begin{equation}
 p(C_1|x) = y(x) = \sigma (w_\top x),\qquad p(C_2|x) = 1-p(C_1|x),
\end{equation}

чтобы обучить эту модель мы можем просто оптимизировать правдоподобие по $w$.

Для набора ${x_n, t_n}$, где $x_n$ -- входы, а $t_n \in \{0;1\}$ -- соответствующие метки классов, получается такое правдоподобие:

\begin{equation}
 p(t|w) = \prod_{n=1}^N y_n^{t_n}(1-y_n)^{1-t_n},\quad\text{где}\quad y_n = p(C_1|x_n).
\end{equation}

И теперь мы, максимизируя логарифм вероятности, ищем наилучшие параметры функции правдоподобия для этого можно использовать различные оптимизаторы.

\begin{equation}
 \lh (w) = -\ln p(t|w) = -\sum_{n=1}^N[t_n \ln y_n + (1-t_n) \ln (1-y_n)].
\end{equation}

Теперь можно легко обобщить задачу на несколько классов. Только вместо сигмоиды будем использовать softmax функцию. Для $K$ классов получаем:

\begin{equation}
 p(C_k|x) = \frac{p(x|C_k)p(C_k)}{\sum_{j=1}^K p(x|C_j)p(C_j)} = \frac{e^{a_k}}{\sum_{j=1}^K e^{a_j}},
\end{equation}

где количество аргументов $a_k = \ln p(x|C_k)p(C_k)$ равняется количеству классов. Функция правдоподобия почти не изменилась. Пусть на вход метки класса подаются в формате one-hot векторов, тогда для набора векторов $T = {t_n}$ функция правдоподобия выглядит следующим образом:

\begin{equation}
 p(T|w_1,\ldots,w_K) = \prod_{n=1}^N \prod_{k=1}^K p(C_k|x_n)^{t_{nk}} = \prod_{n=1}^N \prod_{k=1}^K y_{nk}^{t_{nk}}.
\end{equation}

, где $y_{nk} = y_k(x_n)$. Опять переходим к логарифму и получаем функцию максимального правдоподобия для $K$ классов:

\begin{equation}
 \lh (w_1,\ldots,w_K) = -\ln p(T|w_1,\ldots,w_K) = \sum_{n=1}^N \sum_{k=1}^K t_{nk}\ln y_{nk}.
\end{equation}


\subsection{Метод опорных ввекторов}































\section{ПРАКТИЧЕСКАЯ ЧАСТЬ}


\subsection{Используемые инструменты}

\begin{itemize}
 \item Python 3


 \item NumPy


 \item Pandas


 \item Matplotlib



 \item Pymorphy2


 \item Razdel



 \item TensorFlow



 \item Scikit-learn



 \item JavaScript


 \item HTML и CSS



\end{itemize}














\subsection{Сбор данных}


Самым главным этапом перед созданием моделей-классификаторов является сбор данных, этот процесс включает в себя несколько основных этапов:

\bigskip
\begin{itemize}
 \item обработка и подготовка текстов;
 \item разметка текстов;
 \item обработка полученных результатов.
\end{itemize}

\subsubsection{Обработка и подготовка текстов}

В качестве основного текста для разметки был взят роман Михаила Афанасьевича Булгакова <<Мастер и Маргарита>>.

\bigskip
Обработка документа:

\bigskip
\begin{enumerate}
\item произведение было очищено от нежелательных подстрок регулярными выражениями;
\item разделено на тексты по символу перевода строки <<\textbackslash n>>;
\item из полученных текстов восстановлена прямая речь;
\item тексты содержащие больше 52 слов разделены с использованием библиотеки <<razdel>>.
\end{enumerate}

\bigskip
Формирование заданий:

\bigskip
\begin{itemize}
 \item для разметки выделены тексты от 5 до 52 слов;
 \item для каждого текста определен контекст:  не менее 40 слов перед и не менее 15 после текста.
\end{itemize}



\subsubsection{Разметка текстов}

Чтобы приступить к разметке сначала нужно определить множество меток классов, для этого обратимся к истории создания эмоциональных моделей ведушими профессорами в области изучения эмоций. В 1980 году Роберт Плутчик в своей работе \cite{Plutchik} определил колесо эмоций рис. \ref{fig:plutchik}. Данная модель была взята за основу и дополнена моделью Пола Экмана, которую он описал в работе \cite{Ekman2004} 2004 года и обновил в статье \cite{Ekman2011} 2011 года.

\begin{figure}[ht]
    \centering
    \includegraphics[scale=0.5]{plutchik.png}
    \caption{Колесо эмоции Роберта Плутчика}
    \label{fig:plutchik}
\end{figure}

В результате получилось множество, состоящее из 9 основных эмоций и их производных, дополненное нейтральным классом:

\bigskip
\begin{itemize}
\item \textbf{Злость} (anger) --- желание выразить агрессию или причинить зло, общая для обеих моделей.\\
Может проявляться в словах, мимике, поступках.
Примеры: злость на оскорбление, на несправедливость, злость на плохое отношение.\\
Гнев --- более интенсивная эмоция, досада --- менее.

\item \textbf{Интерес} (anticipation) --- предчувствие важного события, только в модели Плутчика.\\
Проявляется в нетерпении, волнении.
Примеры: ожидание праздника, ожидание начала каникул, ожидание плохой оценки.\\
Настороженность --- более интенсивная, предвкушение --- менее.

\item \textbf{Радость} (joy) --- чувство удовольствия, весёлого настроения и счастья, общая для обеих моделей.\\
Проявляется в смехе, улыбке, ласковом обращении к другим.
Примеры: радость по поводу подарка, общения с другом.\\
Восторг --- более интенсивная, безмятежность --- менее.

\item \textbf{Доверие} (trust) --- ​открытое теплое отношение к чему бы то ни было (другу/животному/миру/...), только в модели Плутчика.\\
Проявляется в уверенности в положительном исходе.
Примеры: доверие другу при встрече с неожиданностями, доверие к собаке, что не укусит, доверие к доктору, что он делает полезные вещи.\\
Восхищение --- более интенсивная, принятие --- менее.

\item \textbf{Страх} (fear) --- состояние перед реальным или предполагаемым бедствием, общая для обеих моделей.\\
Проявляется в волнении, напряжении.
Пример: страх наказания, страх проигрыша, страх попасть в аварию.\\
Ужас --- более интенсивная, тревога --- менее.

\item \textbf{Удивление} (surprise) --- эмоциональная реакция на неожиданную ситуацию, общая для обеих моделей.\\
Удивление может проявляться в хороших и плохих ситуациях.
Примеры: получил плохой отзыв на работу вместо ожидаемого хорошего, директор школы привел в класс собаку, одноклассник вырос на 10 см за лето.\\
Изумление --- более интенсивная, возбуждение --- менее.


\item \textbf{Грусть} (sadness) --- отсутствие радости,​ неудовлетворенность происходящим, отстраненность, общая для обеих моделей.\\
Проявляется в нежелании веселиться с другими, желании заботы и участия.
Примеры: мама уехала в командировку надолго, не покупают собаку или велосипед, никак не дается математика.\\
Горе --- более интенсивная, печаль --- менее.

\item \textbf{Неудовольствие} (disgust) --- эмоциональная реакция на неприятную ситуацию или объект.\\
Проявляется в неприятии человека, любых вещей, ситуаций, общая для обеих моделей.
Примеры: когда сталкиваешься с неприятным запахом, грязными вещами, плохим поведением.\\
Отвращение --- более интенсивная, скука --- менее.

\item \textbf{Презрение} (contempt) --- пренебрежительное отношение к кому-чему-нибудь морально низкому, недостойному, подлому. Презрение связано с чувством превосходства. Также оно может перейти в безразличное отношение к кому-чему-то. Только в модели Экмана.

\item \textbf{Нейтральное} (neutral) --- безэмоциональное повествование.

\end{itemize}

\bigskip
Разметка осуществлялась с помощью краудсорсинговой платформы <<Яндекс.Толока>>.

\begin{definition}
 Краудсорсинг --- это привлечение добровольцев и экспетртов для выполнения определенной работы, действующих на добровольной или комерческой основе.
\end{definition}

Для получения более точной разметки данных были использованы встроенные методы и инструменты контроля качества:

\bigskip
\begin{itemize}
 \item график времени выполнения страницы заданий рис. \ref{fig:task-time} нужен, чтобы видеть на сколько вдумчиво эксперт расставляет метки;
 \item график выполнения заданий рис. \ref{fig:task-accept} показывает сколько выполнено страниц заданий, сколько просрочено и сколько пропущено. По нему можно судить о сложности выполнения задания и использовать эту информацию в процессе формирования новых пулов заданий;
 \item сформирована система правил, которая позволяет контролировать процесс разметки в автономном режиме:
    \medskip
    \begin{itemize}
     \item Eсли пропущенных подряд страниц заданий $\geqslant 10$, то заблокировать на проекте на $2$ дня;
     \item Eсли отправленных страниц заданий $\geqslant 50$, то заблокировать на проекте на $2$ дня;
     \item Минимальное время на страницу заданий --- $250$ сек. Учитывать последних страниц заданий --- $15$. Если количество ответов $\geqslant 5$ и количество быстрых ответов $\geqslant 5$, то заблокировать на проекте на $2$ дня и т.д.
    \end{itemize}
 \item размечены контрольные задания, с их помощью можно отслеживать примерную точность (accuracy), как всего набора данных, так и набора, полученного от одного эксперта;
 \item каждое задание выполняло три различных эксперта (перекрытие x3);
 \item агрегация результатов производилась методом Дэвида-Скина. Он автоматически оценивает для каждого исполнителя $|L|^2$ параметров, где $L$ --- множество возможных различных значений для агрегации и возвращает итоговый ответ и его статистическую значимость.
\end{itemize}


\begin{figure}[ht]
    \centering
    \includegraphics[scale=0.5]{task-time.png}
    \caption{Время выполнения страницы заданий (детализация по 5 минут)}
    \label{fig:task-time}
\end{figure}



\begin{figure}[ht]
    \centering
    \includegraphics[scale=0.5]{task-accept.png}
    \caption{Выполнение страниц заданий (детализация по 5 минут)}
    \label{fig:task-accept}
\end{figure}

\bigskip
Была разработана форма задания рис. \ref{fig:task}. Каждое задание состоит из трех текстовых блоков. Страница заданий состоит из двух не размеченных заданий и одного контрольного, такое разбиение оптимально для получения качественных результатов. Эксперт должен прочитать каждый текстовый блок и отметить эмоции, которые, по его мнению, описаны в выделенном фрагменте.

\begin{figure}[ht]
    \centering
    \includegraphics[scale=0.5]{task.png}
    \caption{Форма задания в сервисе <<Яндекс.Толока>>}
    \label{fig:task}
\end{figure}

\bigskip
В результате был сформирован набор данных с таким распределением классов рис. \ref{fig:class_distribution}.



\begin{figure}[ht]
    \centering
    \includegraphics[scale=0.45]{class_distribution.png}
    \caption{Распределение классов в итоговом наборе данных}
    \label{fig:class_distribution}
\end{figure}

\subsection{Модели классификации}

\subsubsection{Предобработка текстов}

Чтобы работать с текстами, сначала их нужно нормализовать. Этот процесс включает несколько этапов.

\bigskip
\begin{enumerate}
 \item приводим текст к нижнему регистру;
 \item удаляем все <<не слова>> и <<стоп-слова>>;
 \item лемматизируем текст.
\end{enumerate}

\bigskip
Вот пример обработки небольшого текста:

\bigskip
\fbox{Квартира простояла пустой и запечатанной только неделю.} $\to$ \fbox{квартира простаивать пустой запечатывать неделя}


\subsubsection{Представление предложений}


Теперь текст нужно перевести в векторное пространство $R^n$, где $n$ --- размерность признаково пространства используемой модели. Пусть текст $D$ состоит из слов $d \in D$, $f$ --- модель, строящая отображение пространства слов в векторное пространство действительных чисел $f(d) \to R^n$. Тогда текст для классификатора выглядит так:

\begin{equation*}
 \frac{1}{\#D}\sum_{d \in D} f(d) \in R^n
\end{equation*}

В этой работе использованы модели предобученные на корпусе русскоязычных текстов <<Тайга>>:

\bigskip
\begin{itemize}
 \item word2vec \& skip-gram: \textit{tayga\_upos\_skipgram\_300\_2\_2019} ($n = 300$);
 \item ELMo: \textit{tayga\_lemmas\_elmo\_2048\_2019} ($n = 2048$).
\end{itemize}

\bigskip
Особенность применения ELMo заключается в том, что берется среднее значение всех слоев для каждого слова.

\subsection{Архитектура моделей классификации}

Для классификации были выбраны алгоритмы классического машинного обучения:

\bigskip
\begin{itemize}
 \item случайный лес (Random Forest);
 \item логистическая регрессия (Logistic Regression);
 \item метод опорных векторов (Support Vector Machine).
\end{itemize}

\bigskip
Схематично модели классификации представлены на рис. \ref{fig:models}.

\begin{figure}[ht]
    \centering
    \includegraphics[scale=0.6]{models.png}
    \caption{Архитектура моделей классификации}
    \label{fig:models}
\end{figure}


\subsection{Эксперименты}


Для эмоциональной модели Роберта Плутчика результаты получились следующие:

\bigskip
\begin{table}[ht]
\caption{Значения Macro F1 меры}
\label{tab:plutchik}
\centering
\begin{tabular}{ccccc}
Macro F1 & \textbf{log reg} & \textbf{SVM} & \textbf{random forest} & \textbf{tuned random forest}\\
\textbf{w2v} & 0.21519548 & 0.27992996 & 0.23415391 & 0.23932876\\
\textbf{ELMO} & 0.25900211 & 0.32828541 & 0.31197309 & 0.43861588
\end{tabular}
\end{table}


Для эмоциональной модели Пола Экмана:

\bigskip
\begin{table}[ht]
\caption{Значения Macro F1 меры}
\label{tab:ekman}
\centering
\begin{tabular}{ccccc}
Macro F1 & \textbf{log reg} & \textbf{SVM} & \textbf{random forest} & \textbf{tuned random forest}\\
\textbf{w2v} & 0.25493596 & 0.18245115 & 0.21223037 & 0.37710993 \\
\textbf{ELMO}& 0.20242784 & 0.3007757 & 0.34183484 & 0.42405119
\end{tabular}
\end{table}









































% \label{chapter:OptimalBeamDesign}
% \thispagestyle{fancy}
%
%
% \par
%
%
%
% Наконец, в \,\ref{section:NumericalMethod:AuxiliaryProblem}, \ref{section:NumericalMethod:EVP}
% обсуждаются численные аспекты решения задачи о свободных колебаниях балки.
% Предлагается численный алгоритм градиентного типа решения задачи.
%
%
%
% Приводятся и анализируются полученные результаты вычислительных экспериментов.
%
%
%
% \bigskip
% \par
% Для практической реализации использовались:
% \begin{enumerate}
% \item Язык Python 3.7.2
% \item Библиотеки расширения для Python: matplotlib 3.0.3, numpy 1.16.1.
% \item Среда разработки visual studio code 1.34.1
% \item PyQt 5.12.1 используется для графической состовляющей модуля
% \end{enumerate}
%
%
%
%
% \par
% Библиотека "numpy" - это библиотека языка Python, добавляющая поддержку больших многомерных массивов и матриц, а так же большую библиотеку высокоуровневых математических функций для операций с этими массивами. Используется как необходимая библиотека для matplotlib
%
%
%
% \bigskip
% \par
% Библиотека "matplotlib" - это библиотека двумерной графики для языка програмирования Python, с помощью которой можно создавать высококачественные рисункиразличных форматов. Используется вывода графиков.
%
%
%
% \bigskip
% \par
% PyQt --- набор «привязок» графического фреймворка Qt для языка программирования Python, выполненный в виде расширения Python.
%
%
%
% \section{Численный метод решения краевой задачи для уравнения $(ay'')'' = f$}
\label{section:NumericalMethod:AuxiliaryProblem}
%
%
%
\begin{equation}
\label{DE:4Order:F}
(ay'')'' = f
\end{equation}
%
%
%
\par
Для численного решения задачи
\eqref{DE:4Order:F}
с соответствующими краевыми условиями
будем использовать метод \textbf{\emph{матричной прогонки}}.
%
%
%
Для этого уравнение
\eqref{DE:4Order:F}
представим в виде следующей системы дифференциальных уравнений второго порядка:
\begin{equation}
\begin{cases}
y''(x) = \displaystyle \frac{z(x)}{a(x)},
\\
z''(x) = f(x),
\end{cases}
\qquad
x \in I.
\end{equation}
%
%
%
%
%
\par
Введем на отрезке
\(0 \le x \le 1\)
разностную сетку с постоянным шагом \(H\).
%
%
%
Узлы этой сетки будем обозначать через
\(x_k = kH\),
где
\(k = 0, 1, \ldots, N\).
%
%
%
В свою очередь,
значения функций
\(y(x), z(x), a(x), f(x)\)
в узлах
будем обозначать через
\(y_k, z_k, a_k, f_k\)
соответственно.
%
%
%
 Функцию \(y(x)\) будем аппроксимировать разложив в ряд Тейлора слева \(y(x-h)\) и справа \(y(x+h)\) следующим видом
	\begin{equation*}
	y(x+h)=y(x) + \frac{y'(x)h}{1!} + \frac{y''(x)h^2}{2!} + \frac{y'''(x)h^3}{3!} + \ldots
	\end{equation*}
	\begin{equation}
	y(x-h)=y(x) - \frac{y'(x)h}{1!} + \frac{y''(x)h^2}{2!} - \frac{y'''(x)h^3}{3!} + \ldots
	\end{equation}
	На выходе получаем следующее
	\begin{equation*}
	y(x+h)+y(x-h) = 2y(x)+y''(x)h^2 \quad \Rightarrow
	\end{equation*}
	\begin{equation}
	\frac{y_{k+1}-2y_k+y_{k-1}}{h^2}=y''(x_k)=\frac{z_k}{U_k}
	\end{equation}
	Аналогично выполняем аппроксимацию \(z(x)\)
	\begin{equation}
	\frac{z_{k+1}-2z_k+z_{k-1}}{h^2}=f(k)
	\end{equation}
	Для дальнейшей матричной прогонки определим вектор
	\(X_k= \left( \begin{matrix}
	y_k\\z_k
	\end{matrix} \right)\) 
	и коэффициенты \(A_k,B_k,C_k\).
	\begin{equation}
	A_kX_{k+1}-B_kX_k+C_kX_{k-1}=F_k, \quad k=1\ldots N-1
	\end{equation}
	Выпишем полученную систему уравнений:
	\begin{equation}
	\begin{cases}
	y_{k+1}-2y_k+y_{k-1} - \frac{z_kh^2}{U_k}=0 \\
	z_{k+1}-2z_k+z_{k-1}=f(k)h^2 \\
	F_k= \left( \begin{matrix}
	0\\f(k)h^2
	\end{matrix} \right)
	\end{cases}
	\end{equation}
	Выразим матричное уравнение и выразим коэффициенты \(A_k,B_k,C_k\)
	\begin{equation*}
	\left( \begin{matrix}
	1&0\\0&1
	\end{matrix} \right)X_{k+1}-\left( \begin{matrix}2&\frac{h^2}{U_k}\\0&2\end{matrix}\right)X_k+
	\left( \begin{matrix}1&0\\0&1\end{matrix}\right)X_{k-1}=F_k
	\end{equation*}
	\begin{equation}\label{eq4}
	A_k=\left( \begin{matrix}1&0\\0&1\end{matrix} \right) \quad 
	B_k=\left( \begin{matrix}2&\frac{h^2}{U_k}\\0&2\end{matrix} \right) \quad 
	C_k=\left( \begin{matrix}1&0\\0&1\end{matrix} \right), \quad  A_k=C_k=E
	\end{equation}
	Для определения коэффициентов \(X_k\)~- необходимо рассмотреть обратный итерационный процесс
	\begin{equation}\label{eq5}
	X_k=L_{k+1}X_{k+1}+M_{k+1}
	\end{equation}
	Выведем формулы для коэффициентов \(L_k\) и \(M_k\)
	\begin{equation*}
	A_kX_{k+1}-B_kX_k+C_k(L_kX_k+M_k)=F_k
	\end{equation*}
	\begin{equation*}
	A_kX_{k+1}+X_k(C_kL_k-B_k)=F_k-C_kM_k
	\end{equation*}
	\begin{equation*}
	X_k=-(C_kL_k-B_k)^{-1}A_kX_{k+1}+(C_kL_k-B_k)^{-1}(F_k-C_kM_k)
	\end{equation*}
	\begin{equation}\label{eq6}
	L_{k+1}=-(C_kL_k-B_k)^{-1}A_k
	\end{equation}
	\begin{equation}\label{eq7}
	M_{k+1}=(C_kL_k-B_k)^{-1}(F_k-C_kM_k)
	\end{equation}
	Рассмотрим алгоритм метода матричной прогонки для трех возможных способов закрепления балки.
	
	\bigskip\quad Алгоритм:
	\begin{enumerate}
		\item Вычисляем коэффициенты \(A_k,B_k,C_k\) по формулам (\ref{eq4}) для\\ k=1\ldots N-1
		\item Определяем граничные условия метода в соответствии с типом закрепления балки:
		\begin{enumerate}
			\item Балка свободно оперта на обоих концах свей длины.
			
			В данном случае краевыми условиями на обои концах \((x=0,x=l)\) являются следующие условия:
			\begin{equation*}
			y=y''=0|_{x=0}, \quad y=y''=0|_{x=l}
			\end{equation*}
			Исходя из этого, получим:
			\begin{equation*}
			L_0=\left( \begin{matrix}0&0\\0&0\end{matrix} \right),\quad
			M_0=\left( \begin{matrix}0\\0\end{matrix} \right),
			\end{equation*}
			\begin{equation*}
			L_n=\left( \begin{matrix}0&0\\0&0\end{matrix} \right),\quad
			M_n=\left( \begin{matrix}0\\0\end{matrix} \right),\quad,
			X_n=\left( \begin{matrix}0\\0\end{matrix} \right)
			\end{equation*}
			\item Балка жестко закреплена на обоих концах.\\
		
			
			В данном случае краевыми условиями на обоих концах \\\((x=0,x=l)\) являются следущие условия:
			\begin{equation*}
			y=y'=0|_{x=0}, \quad y=y'=0|_{x=l}
			\end{equation*}
			Исходя из этого, получим:
			\begin{equation*}
			L_0=\left( \begin{matrix}0&0\\\frac{2U_0}{h^2}&0\end{matrix} \right),\quad
			M_0=\left( \begin{matrix}0\\0\end{matrix} \right),
			\end{equation*}
			Вычислим \(X_n\):
			\begin{equation*}
			X_{n-1}=L_nX_n+M_n,\quad y_{n-1}=\frac{h^2z_n}{2U_0};
			\end{equation*}
			\begin{equation*}
			X_{n-1}=\left(\begin{matrix}0&\frac{h^2}{2U_0}\\0&0\end{matrix}\right)\left(\begin{matrix}y_n\\z_n\end{matrix}\right)+\left(\begin{matrix}0\\0\end{matrix}\right);
			\end{equation*}
			\begin{equation*}
			\left(\begin{matrix}l_{11}&l_{12}\\l_{21}&l_{22}\end{matrix}\right)\left(\begin{matrix}y_n\\z_n\end{matrix}\right)+\left(\begin{matrix}m_1\\m_2\end{matrix}\right)=\left(\begin{matrix}0&\frac{h^2}{2U_0}\\0&0\end{matrix}\right)\left(\begin{matrix}y_n\\z_n\end{matrix}\right);
			\end{equation*}
			\begin{equation*}
			\left(\begin{matrix}l_{11}&l_{12}-\frac{h^2}{2U_0}\\l_{21}&l_{22}\end{matrix}\right)\left(\begin{matrix}y_n\\z_n\end{matrix}\right)=\left(\begin{matrix}m_1\\m_2\end{matrix}\right);
			\end{equation*}
			\begin{equation*}
			X_n=-\left(\begin{matrix}l_{11}&l_{12}-\frac{h^2}{2U_0}\\l_{21}&l_{22}\end{matrix}\right)^{-1}\left(\begin{matrix}m_1\\m_2\end{matrix}\right);
			\end{equation*}
			\item Балка жестко закреплена на одном конце и оперта на другом.
					
			В данном случае краевыми условиями на обои концах \((x=0,x=l)\) являются следующие условия:
			\begin{equation*}
			y=y'=0|_{x=0}, \quad y=y'=0|_{x=l}
			\end{equation*}
			Исходя из этого, получим:
			\begin{equation*}
			L_0=\left( \begin{matrix}0&0\\\frac{2U_0}{h^2}&0\end{matrix} \right),\quad
			M_0=\left( \begin{matrix}0\\0\end{matrix} \right),
			\end{equation*}
			\begin{equation*}
			X_N=\left( \begin{matrix}0\\0\end{matrix} \right).
			\end{equation*}
		\end{enumerate}
		\item Вычисляем  коэффициенты \(L_k,M_k\) по формулам (\ref{eq6}) и (\ref{eq7}) для\\ k=1\ldots N-1.
		\item Начиная с \(k=N-1~\text{до}~k=0~\text{вычисляем векторы}~X_k\) по формуле (\ref{eq5}). 
		\item Получив значения \(X-k\), получаем значения \(y_k\text{ и } z_k\).
	\end{enumerate}

%
% Численный метод определения собственной пары %%%
% \section{Численный метод решения задачи о поперечных колебаниях балки}
\label{section:NumericalMethod:EVP}

\begin{equation*}  
(Uy'')''=\lambda Vy
\end{equation*}
Полученное решение краевой задачи  \(y_{(n)}\) используем для определения следующего приближения безрезонансного интервала:
\begin{equation*}  
(Uy_{(n+1)}'')''=f=Vy_{(n)}
\end{equation*}
%
%
%
%
Вариационное представление
\eqref{FirstEigenvalue:VariationalPrinciple}
первого собственного значения $\lambda_1[u]$
определяем следущим \emph{\textbf{отношением Рэлея}}:
%
%
%
%
\begin{equation*}  
\lambda_1^{(n)}=\frac{\int_{0}^{1}Uy_{(n)}''^2dx}{\int_{0}^{1}Vy_{(n)}^2dx},
\end{equation*}
\begin{equation*}  
Uy_{(n)}''^2=\frac{1}{U}(Uy_{(n)}''^2)^2=\frac{1}{U}z^2,
\end{equation*}
\begin{equation}\label{eq9}   
\lambda_1^{(n)}=\frac{\int_{0}^{1}\frac{1}{U}z_{(n)}^2dx}{\int_{0}^{1}Vy_{(n)}^2dx}.
\end{equation}


Решение уравнения (\ref{eq9}) находим методом трапеций\\
\begin{equation}\label{eq10} 
\int_a^bf(x)\approx\frac{h}{2}(f(x_0)+2\sum\limits_{i=1}^{n-1} f(x_i) +f(x_n),
\end{equation}
\begin{equation*}  
h=\frac{b-a}{n}=1 \quad  x_i=a+ih \quad  i=0,1\ldots n,
\end{equation*}
\begin{center}  
\(i=0 \quad x_0+0*1\quad \Rightarrow\quad f(x_0)=Vy^2\bigg|_0\)\\
\(i=1 \quad x_1+1*1\quad \Rightarrow\quad f(x_1)=Vy^2\bigg|_1\)\\
\ldots\\
\ldots\\
\(i=n \quad x_n+1*n\quad \Rightarrow\quad f(x_n)=Vy^2\bigg|_n\)
\end{center}
\par
Далее заносим значения под знак суммы (\ref{eq10}) и находим все собственные значения $\lambda_1[u]$ описывающее резонансное поведение в процессе колебания балки, в этом случае следующая погрешность и является показательной
\begin{equation*}  
|\lambda_1^{(n+1)} + \lambda_1^{(n)}|<\varepsilon\approx 10^{10}.
\end{equation*}
%
% Поиск седловой точки %%%
% \section{Поиск седловой точки}


\par
Оптимальное решение $w$ и соответствующая ему собственная функция $y_1$ образуют седловую точку функционала 
\begin{equation}
F(u,z) = \frac{\int_{0}^{1}eu^{\nu}z''^2  \, dx}{\int_{0}^{1}\rho uz^2 \, dx},
\end{equation}
т.\,е. выполняются неравенства
\begin{equation}
F(u,y_1) \leq F(w,y_1) \leq F(w,z),
\end{equation}
где $u$ и $z$ пробегают допустимые множества.
%
\bigskip
\par
Седловую точку ищем с помощью следущей процедуры: 
%
\par 
\begin{enumerate}
	\item задается начальное приближение $u_0 \in U$;
	%
	\item  для заданного $u_n$ решается задача на собственные значения для определения собственной пары ($\lambda_1[u_n], y_1[u_n]$);
	%Это эквивалентно решению задачи на собственные значения.
	
	\par
	%
	\item находится новое приближение $u_{n+1}$, для которого
	\begin{equation}
	F(u_{n+1},y_1[u_n]) = \max_{u \in U} F(u,y_1[u_n]);
	\end{equation}
	
	\item если 
	\begin{equation}
	|\lambda_1[u_{n+1}] - \lambda_1[u_{n}]| < \varepsilon,
	\end{equation}
	где $\varepsilon$~--- заданная  точность, то итерационная процедура завершается, а $u_{n+1}$ выступает в качестве оптимального распределения.
	\bigskip
	\par
	В противном случае переходим к шагу 2 с $n=n+1$.
\end{enumerate}

\bigskip
\par
После определения собственной пары составляем уравнение максимизации колебательного процесса упругой балки, для определения нового оптимального приближения, с последующим определением новой собственной пары, для продолжения итерационного процесса
\begin{equation}
\label{MaximizingVibrating} 
e\xi y''^2_n - \lambda\rho\xi y^2_n + \mu\xi \rightarrow \max,
\end{equation}

\begin{equation*}
\begin{aligned}
\int_0^1{eU_{n+1}y''^2_ndx} - \lambda_n\int_0^1{\rho U_{n+1}y^2_ndx}\geqslant	\\
\geqslant \int_0^1{eU_ny''^2_ndx} - \lambda_n\int_0^1{\rho U_ny^2_ndx}\\
\end{aligned}
\end{equation*}

\begin{equation*} 
e(x)\xi y''^2_n - \lambda\rho(x)\xi y^2_n + \mu\xi \rightarrow \max.
\end{equation*}
%
%
%
Подставляем $\mu$  и для каждого x, решим задачу получив $\xi(x)$,
%
%
%
\begin{equation*} 
\xi(e(x) y''^2_n - \lambda\rho(x)\xi y^2_n + \mu) \rightarrow \max,
\end{equation*}
$\mu$ надо подобрать так, чтоб интеграл $\int_0^1{\xi(x)dx} = 1$ был равен единице, вводим следующую функцию: 
\begin{equation*} 
f(\mu)=\int_0^1{\xi_{\mu}(x)dx} - 1,
\end{equation*}
\begin{equation*} 
\xi \in [\alpha,\beta] \qquad \xi(x) \rightarrow \max,
\end{equation*}
%
%
%
где $\alpha , \beta$~--- ограничения на площадь, $\alpha= u_{min}$, $\beta= u_{max}$ 
%
%
%
\begin{center}
$$ (\lambda_n,y_n) \rightarrow \xi(x)= U_{n+1},$$
$$ (\lambda_{n+1},y_{n+1}) \rightarrow \xi(x)= U_{n+2},$$
\ldots \\
\ldots \\
$$(\lambda_{n+k-1},y_{n+k-1}) \rightarrow \xi(x)= U_{n+k};$$
\end{center}
%
%
%
%
Для решения оптимизационной поточечной задачи, использем метод секущих Ньютона (не требующий производную функции) находим ноль функции, и задача \eqref{MaximizingVibrating} принимает следущий вид
%
%
%
%
\begin{equation*} 
f(\mu)=\int_0^1{\xi_{\mu}(\frac{ez^2_n}{u^2\rho^2}+\lambda\rho y^2_n+\mu)dx} - 1
\end{equation*}
%
%
метод секущих для следующей итерации $x_n$
%
%
\begin{equation*} 
x_n=x_{n-1} - f(x_n -1) \frac{x_{n-1}-x_{n-2}}{f(x_{n-1})-f(x_{n-2})}=\frac{x_{n-2}f(x_n-1)-x_{n-1}f(x_{n-2})}{f(x_{n-1}-f(x_{n-2}))}.
\end{equation*}
Для первого выполнения итерации берем  $x_{n-1}$~--- значения первичной собственной пары $x_{n-2} = 0$. Задача на собственные значения решается многократно, останавливается когда $|\lambda_{n+1}-\lambda_n| <\varepsilon\approx 10^{5}$

%О том как ищешь седловую точку.

%Поточечная задача оптимизации от $\xi$ и т.д.

%Как решаешь эту задачу.

%
% РЕЗУЛЬТАТЫ ВЫЧИСЛИТЕЛЬНОГО ЭКСПЕРИМЕНТА %%%
% \section{Результаты вычислительного эксперимента}
%
%
%
%
\par
Примененные численные методы демонстрировали сходимость к оптимальному решению балки, с определением формы балки и частоты безрезонансных колебаний балки, с учетом заданных параметров проектирования (толщины, массы), при ограничении на толщину.
%
%
%
\begin{enumerate}
	\item 	$h(x) = 2\sqrt{x}$; \;\; $u_{min} = 0{,}4$; \;\; $u_{max} = 1{,}8$
	\item 	$h(x) = 181.263 \rho^2e/x^6 $; \;\; $u_{min} = 0{,}4$; \;\; $u_{max} = 1{,}8$
\end{enumerate}

\begin{center}
	\begin{figure}
		\includegraphics[width=1\linewidth]{untitled2.png}
		\caption{Оптимальное распределение толщины балки $u_{opt}$ (1).}
		\bigskip
		\includegraphics[width=1\linewidth]{untitled.png}
		\caption{Оптимальное распределение толщины балки $u_{opt}$ (2).}
	\end{figure}
	%
	%
	%
\end{center}



%Картинки.

%При каких параметрах получены.

%Каков эффект оптимизации.

%Анализ.

%Параметры $\alpha, \beta, \nu$.

%Оптимальное распределение
%Оптимизация эффективность




%\includegraphics[width=0.5\linewidth]{1code.png}
%\includegraphics[width=0.5\linewidth]{2code.png}
%\\
%При каких параметрах получены.
%
%Каков эффект оптимизации.
%
%Анализ.
%
%Параметры $\alpha, \beta, \nu$.

\chapter*{ЗАКЛЮЧЕНИЕ}
\addcontentsline{toc}{chapter}{ЗАКЛЮЧЕНИЕ}

В ходе проделанной работы был сформирован оригинальный набор размеченных данных, содержащих эмотивную лексику. На основе знаний полученных из рассмотренной литературы выбрано несколько архитектур моделей классификации. Был подробно разобран и обоснован их математический аппарат и написана реализация на языке Python c использованием библиотек машинного обучения и предобученных семантических моделей русского языка.

\bigskip
Был проведен сравнительный анализ результатов точности классификации метода опорных векторов (SVM), логистической регрессии (logistic regression) и случайного леса (random forest) в связке с распределенными представлениями слов: word2vec и ELMo.

\bigskip
Согласно результатам исследования лучшие метрики показала модель, основанная на случайном лесе с решающими деревьями и ELMo. Из этого следует вывод, что для задачи сентимент-анализа использование более сложных и тяжеловесных контекстуализированных эмбеддингов --- хорошее решение, т.к. они лучше обобщают сравнительно небольшие тексты и позволяют классификатору лучше разделить выборку. Но качество Macro F1 меры все равно не достигло высоких показателей. Возможно это связано с недостаточной величиной собранного набора данных в результате чего того  количества признаков, которые выделила модель, не хватило для качественной классификации.

\bigskip
Возможными направлениями для дальнейших исследований могут стать модели для выделения именованных сущностей. Определение текстов относящихся именно к этим сущностям и реализация сентимент-анализа с использованием информации от этих моделей в более комплексных архитектурах с механизмом внимания. Также увеличение набора данных поможет улучшить результаты.
































% \begin{enumerate}
% 	\item
% 	Исследована задача максимизации фундаментальной частоты свободных колебаний балки, решение которой позволяет находить оптимальные формы балки, при которых она наименее подвержена резонансу.
%
%
%
% 	Поэтому результаты данной работы могут найти практическое применение.
%
%
%
% 	\item
% 	В работе сформулирована итерационная процедура поиска седловой точки, позволяющая решить задачу максимизации фундаментальной частоты свободных колебаний балки.
%
%
%
% 	Этот подход к решению данной задачи ранее в литературе не применялся и поэтому является новым.
%
%
%
% 	Поэтому можно ожидать, что предложенный метод позволяет строить сходящиеся к оптимальному решению последовательности.
%
%
%
% 	\item
% 	Для этой итерационной процедуры был реализован численный метод в виде библиотеки на языке програмирования Python, а так же проведены вычислительные эксперементы, в ходе которых последовательность приближений демонстрировала сходимость.
%
%
%
% \end{enumerate}
%
%

%%%%%%%%%%%%%%%%%%%%%%%%%%%%%%%%%%%%%%%%%%%%%%%%%%%
% https://tex.stackexchange.com/questions/179691/removing-underline-from-journal-title-when-using-hyperref
{\normalem % remove underline
\printbibliography[heading=bibintoc,title={СПИСОК ИСПОЛЬЗОВАННЫХ ИСТОЧНИКОВ}]
}
%%%%%%%%%%%%%%%%%%%%%%%%%%%%%%%%%%%%%%%%%%%%%%%%%%%
{\CenterChapterHeading\chapter*{ПРИЛОЖЕНИЯ}
\addcontentsline{toc}{chapter}{ПРИЛОЖЕНИЯ}
\newpage
}
\chapter*{Приложение}
\addcontentsline{toc}{chapter}{Приложение}
\label{chapter:Supplement}



\phantom{x}
\newpage
\phantom{x}
\newpage
\phantom{x}
\newpage
\phantom{x}

%%%%%%%%%%%%%%%%%%%%%%%%%%%%%%%%%%%%%%%%%%%%%%%%%%%

pic example
\begin{figure}[ht]
\centering
    \begin{subfigure}[b]{0.3\textwidth}
    \centering
        $$\begin{array}{l}
        F \to x \;|\; y \;|\; (S) \\
        T \to F \;|\; T \ast F \\
        S \to T \;|\; S + T \\
        \end{array}$$
        \caption{}
    \end{subfigure} %
    \begin{subfigure}[b]{0.6\textwidth}
    \centering
        \includegraphics[scale=0.5]{look-track.png}
        \caption{}
    \end{subfigure}

    \caption{(a) Арифметические выражения;
             (б) Look R96 2016.}
    \label{fig_parsetree}
\end{figure}


% table example
\begin{table}[ht]
\caption{Расчет параметров}
\label{tab_weight}
\centering
    \begin{tabular}{|c|c|c|c|c|c|c|c|c|}
    \hline \multirow{2}{*}{Параметр $x_i$} & \multicolumn{4}{c|}{Параметр $x_j$} &
        \multicolumn{2}{c|}{Первый шаг} & \multicolumn{2}{c|}{Второй шаг} \\
    \cline{2-9} & $X_1$ & $X_2$ & $X_3$ & $X_4$ & $w_i$ &
        ${K_\text{в}}_i$ & $w_i$ & ${K_\text{в}}_i$ \\
    \hline $X_1$ & 1 & 1 & 1.5 & 1.5 & 5 & 0.31 & 19 & 0.32 \\
    \hline $X_2$ & 1 & 1 & 1.5 & 1.5 & 5 & 0.31 & 19 & 0.32 \\
    \hline $X_3$ & 0.5 & 0.5 & 1 & 0.5 & 2.5 & 0.16 & 9.25 & 0.16 \\
    \hline $X_4$ & 0.5 & 0.5 & 1.5 & 1 & 3.5 & 0.22 & 12.25 & 0.20 \\
    \hline \multicolumn{5}{|c|}{Итого:} & 16 & 1 & 59.5 & 1 \\
    \hline
    \end{tabular}
\end{table}

\end{document}
