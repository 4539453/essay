\chapter*{РЕФЕРАТ}

% Использование нейронных сетей и методов машинного обучения для сентимент-анализа текстов

% Введение
%   обосновывается актуальность ВКР и практическая значимость
%   объект, предмет, цель и задачи ВКР
%   определяются методы исследования
%   краткий обзор информационной базы исследования.

% Теоретическая часть
%   Обзор предметной области (включая краткий обзор этих исследования)
%       По количеству классов
%       По подходам к анализу
%   Задача классификации и сентимент-анализа текстов

%   Подготовка данных?

%   Обработка текстов
%       Основная идея архитектуры нейронных сетей
%           Функции активации
%       Дистрибутивная семантика
%       Распределенные представления слов word2vec
%       ELMo

%   Классификация
%       Macro F1
%       Cross validation
%       log reg
%       SVM
%       random forest

% Практическая часть
%   Используемые иструменты
%   Сбор данных
%   Модели классификации
%       1
%       2
%   Эксперименты

% Заключение
%   общие результаты ВКР
%   обобщённые выводы и предложения, указываются
%    перспективы применения результатов на практике
%    возможности дальнейшего исследования проблемы

%
% Список источников
%
% Приложение
%   Toloka
%   Models

\bigskip\par
Выпускная квалификационная работа содержит \pageref*{LastPage}~страниц, \totfig~рисунка,                                        \tottab~таблицу, XX использованных источника.

\bigskip\par
СЕНТИМЕНТ-АНАЛИЗ, АНАЛИЗ ТОНАЛЬНОСТИ, КЛАССИФИКАЦИЯ ТЕКСТОВ, МАШИННОЕ ОБУЧЕНИЕ, РАСПРЕДЕЛЕННЫЕ ПРЕДСТАВЛЕНИЯ СЛОВ, КРАУДСОРСИНГ, НАБОР ДАННЫХ

\bigskip\par
Данная работа посвящена изучению сентимент-анализа русскоязычных художественных текстов с помощью методов распределенного представления слов и машинного обучения. В ходе исследования был сформирован набор данных для обучения получившихся моделей.

\bigskip
Теоретическая часть работы состоит из обзора и анализа существующих исследований в этой области, обоснования математического аппарата применяемых методов, основных подходов в предобработке русскоязычных текстов и постановки задачи сентимент-анализа.


\bigskip
В практической части разобрана проблема формирования качественного набора данных, реализация и применение математически обоснованных моделей классификации на основе алгоритмов машинного обучения в связке с моделями распределенных представлений слов, а также анализ полученных результатов и итоги исследования.

\bigskip
Данная работа представлена на 19-й международной конференции <<Авиация и космонавтика>> \cite{avia} и на международной молодежной научной конференции XLVII <<Гагаринские чтения>> \cite{gagar}.

