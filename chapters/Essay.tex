\chapter*{РЕФЕРАТ}

% Использование нейронных сетей и методов машинного обучения для сентимент-анализа текстов

% Введение
%   обосновывается актуальность ВКР и практическая значимость
%   объект, предмет, цель и задачи ВКР
%   определяются методы исследования
%   краткий обзор информационной базы исследования.

% Теоретическая часть
%   Задача классификации и сентимент-анализа текстов
%   Виды классификации (включая краткий обзор этих исследования)
%       По количеству классов
%       По подходам к анализу

%   Подготовка данных?

%   Обработка данных
%       Основная идея архитектуры нейронных сетей
%           Функции активации
%       Дистрибутивная семантика
%       Распределенные представления слов word2vec
%       ELMo

%   Алгоритмы классического машинного обучения
%       Macro F1
%       Cross validation
%       log reg
%       SVM
%       random forest

% Практическая часть
%   Используемые иструменты
%   Сбор данных
%   Модели классификации
%       1
%       2
%   Эксперименты

% Заключение
%   общие результаты ВКР
%   обобщённые выводы и предложения, указываются
%    перспективы применения результатов на практике
%    возможности дальнейшего исследования проблемы

%
% Список источников
%
% Приложение
%   Toloka
%   Models

\par
Выпускная квалификационная работа содержит \pageref*{LastPage}~страниц, \totfig~рисунка,                                        \tottab~таблицу, XX использованных источника.
\bigskip

\par
КЛЮЧЕВОЕ СЛОВО 1, КЛЮЧЕВОЕ СЛОВО 2, …
\bigskip


\par
Краткое описание содержания работы.

