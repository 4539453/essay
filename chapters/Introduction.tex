\chapter*{Введение}
\addcontentsline{toc}{chapter}{Введение}

\par
Классификация текстов --- невероятно популярная задача. Мы пользуемся текстовыми классификаторами в почтовом клиенте: он классифицирует письма и фильтрует спам. Другие приложения включают классификацию документов, обзоров и так далее.

\bigskip\par
Обычно классификация текстов используется не как самостоятельная задача, а является частью более крупного пайплайна. Например, голосовой помощник классифицирует ваше высказывание, чтобы понять, что вы хотите (например, установить будильник, заказать такси или просто поболтать) и передать ваше сообщение другой модели в зависимости от решения классификатора.  Другой пример --- поисковый движок: он может использовать классификаторы для определения языка запроса, чтобы предсказать его тип (например, информационный, навигационный, транзакционный) и понять хотите ли вы увидеть картинки или видео помимо документов и прочего.

\bigskip\par
Моя работа посвящена одному из приложений классификации --- автоматическому определению эмоциональной оценки русскоязычных текстов. Главная особенность  заключается в том, что предсказание базируется на эмоциональных моделях, предложенных Робертом Плутчиком и Полом Экманом.


\bigskip\par
Поскольку большинство датасетов для классификации содержат только одну правильную метку. А у нас многоклассовая классификация.
