\chapter*{ВВЕДЕНИЕ}
\addcontentsline{toc}{chapter}{ВВЕДЕНИЕ}
\vspace*{-0.5cm}

% Введение
%   обосновывается актуальность ВКР и практическая значимость
%   объект, предмет, цель и задачи ВКР
%   определяются методы исследования
%   краткий обзор информационной базы исследования.

Автоматическая классификация текстов является важной задачей обработки естественного языка. Эта работа посвящена одному из приложений классификации --- автоматическому определению эмоциональной окраски русскоязычных художественных текстов. Главная особенность заключается в том, что предсказание базируется на эмоциональных моделях, предложенных Робертом Плутчиком и Полом Экманом.

\bigskip
Мультиклассификатор может интегрироваться в IoT и другие интеллектуальные устройства, чтобы эти устройства могли действовать на основе обнаруженных эмоциональных состояний пользователей. Может быть использован как интеллектуальный ассистент, например, во время психологических консультаций, чтобы лучше отслеживать и понимать состояние пациента и помогать врачам более эффективно оказывать поддержку. Или как инструмент для исследования как собственных эмоций, так и эмоций окружающих.

\bigskip\noindent
Основными задачами является:

\begin{itemize}
 \item изучение существующих работ в данной области;
 \item формирования набора размеченных данных, содержащих эмотивную лексику;
 \item математическое обоснование моделей классификации и их реализация на языке программирования Python;
 \item анализ полученных результатов.
\end{itemize}

\bigskip
Подавляющее большинство работ на эту тему сводятся к бинарной классификации, модели способны определять только два класса: <<положительный>> и <<отрицательный>> или рассматривается похожая задача, но регрессии, в ней появляется промежуточные классы. И почти не существует моделей мультиклассификации по эмоциональным моделям. Основная проблема заключается в отсутствии данных для исследований. В этой работе описан процесс сбора необходимых данных и оценка качества их классификации.

