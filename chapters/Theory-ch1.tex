\section{Сентимент\nobreak-анализ текстов}

\begin{definition}
 Классификация текста --- использование математических моделей для определения принадлежности текста, на основании его содержимого, к конкретному классу из определенного множества.
\end{definition}

\begin{definition}
 Мультиклассификатор текстов --- алгоритм или обученная модель, предсказывающие принадлежность каждого входного текста к одному или нескольким классам, множество классов определено заранее.
\end{definition}

\begin{definition}
 Сентимент-анализ (анализ тональности) --- автоматическое выявление в текстах эмоционально окрашенной лексики и эмоциональной оценки авторов.
\end{definition}

\begin{definition}
 Сентимент --- совокупность чувств и взглядов, как основа для действия или суждения; общая эмоциональная установка.
\end{definition}

%%% 1. Semina %%%
\par
Сентимент-анализ, как направление компьютерной лингвистики, стал очень популярен последние десятилетие. С появлением больших данных насыщенных эмоциональной составляющей возникла потребность в их обработке. Компании начали устраивать соревнования с внушительными призовыми фондами, а исследователи искать лучшую архитектуру для классификации этих данных. Так в открытом доступе появились большие размеченные датасететы с отзывами, данными из соцсетей и новостями.

\bigskip\par
Сам термин <<сентимент>> означает совокупность чувств и взглядов, как основа для действия или суждения; общая эмоциональная установка. Целью сентимент-анализа является выделении этих тональных компонент из текста. Рассмотрим его применение на разных уровнях.

\bigskip\par
Пусть есть целый документ, тогда, как правило, в нем можно выделить один субъект и один объект, а так же сентимент. Ярким примером такого уровня является отзыв. Здесь автор выступает в качестве субъекта, а предмет отзыва --- в качестве объекта. Это уровень документа.

\bigskip\par
Если документ более сложный, то можно рассматривать его на уровне предложений. В результате можно определить эмоциональную окраску всего документа или предложений по отдельности. Зависит от поставленной задачи.

\bigskip\par
Также анализ бывает на уровне аспектов. Смысл его в том, что эмоциональная установка определяется не для конкретно объекта, а для его отдельных составляющих --- аспектов. Например, для объекта <<компьютер>> можно выделить аспекты <<производительность>>, <<дизайн>>, <<сборка>> т.д., другими словами, к аспектам относится все то, к чему могут быть выражены эмоции. Данная задача очень востребована, потому что позволяет точнее определять отношение автора к объекту, а для некоторых задач это очень важный критерий.

\bigskip\par
Последний вид анализа самый сложный и проводится уровне именованных сущностей (Named Entities). Именованная сущность --- это абстрактный или физический объект, который может быть обозначен собственным именем. Сама по себе задача извлечения именованных сущностей (Named Entity Recognition) не из простых, а вкупе с сентимент-анализом становится действительно комплексным решением.










Исследователи уже достигли внушительных результатов в сентимент-анализе английского






Во многом благодаря выдающимся работам ++, которые помогли




\subsection{Методы, основанные на правилах и словарях}












































% \section{Свободные колебания балки. Модель Бернулли\nobreak-Эйлера}
% \label{section:GoverningProblem}
%
%
%
% \par
% Если на балочную конструкцию наложить деформирующее воздействие,
% приводящее к выходу из состояния равновесия,
% то под действием внутренних сил возникают \textbf{\emph{свободные (или собственные)}} колебания.
%
%
%
% \par
% В настоящей работе
% рассматривается упругая балка модели Бернули-Эйлера.
%
%
%
% Напомним, что в классической теории Бернули-Эйлера не учитывается влияние инерции вращения элемента балки и
% деформаций поперечного сдвига.
%
%
%
% При этом в рамках данных допущений предполагается, что размеры поперечных сечений малы по сравнению с длиной
% балки.
%
%
%
% Вместе с тем расчеты, основанные на данной модели балки, все же оказываются довольно точными.
%
%
%
% \par
% Уравнение поперечных свободных колебаний балки имеет следующий вид
% \cite{book:Timoshenko}:
% \begin{equation}
% \label{BeamVibrating:Dynamic}
% \frac{\partial^2}{\partial x^2} \left\{EI\frac{\partial^2 z(x,t)}{\partial x^2} \right\}
% +
% \rho S \frac{\partial^2 z(x,t)}{\partial t^2}
% =
% 0,
% \qquad
% x \in (0,l).
% \end{equation}
%
% В уравнении \eqref{BeamVibrating:Dynamic}
% $x$~--- пространственная координата,
% $t$~--- время,
% $\rho$~--- удельная плотность материала,
% $E$~--- модуль Юнга материала балки,
% $S$~--- распределение площади поперечного сечения балки,
% $I$~--- геометрический момент инерции поперечного сечения,
% $z(x,t)$~--- функция прогибов при колебаниях в точке $(x, t)$,
% $l$~--- длина балки.
%
%
%
% \par
% В модели Бернулли-Эйлера
% при свободных поперечных колебаниях функция прогибов меняется во времени по гармоническому закону
% \cite{book:Timoshenko}
% \begin{equation}
% \label{HarmonicLaw}
% z_k(x, t) = y_k(x) e^{i \omega_k t}.
% \end{equation}
%
%
%
% Здесь индекс $k$ обозначает номер формы колебаний,
% $\omega_k$~--- собственные частоты, $y_k$~--- распределение прогибов.
%
%
%
% \begin{definition}
% Наименьшая частота собственных колебаний $\omega_1$ называется
% \textbf{\emph{фундаментальной частотой}} колебаний.
% \end{definition}
%
%
%
% \par
% Подставляя выражение
% \eqref{HarmonicLaw} в уравнение
% \eqref{BeamVibrating:Dynamic},
% получаем основное уравнение
% \begin{equation}
% \label{BeamVibrating:Dimensional}
% \frac{d^2}{dx^2} \left\{EI\frac{d^2 y}{dx^2} \right\} = \omega^2 \rho S y,
% \qquad
% x \in (0, l).
% \end{equation}
%
%
%
% Здесь ради удобства опущен индекс $k$.
%
%
%
% \par
% Кроме того, на функцию прогибов
% $y$
% налагается система ограничений в виде краевых условий,
% которая определяет тип закрепления балки.
%
%
%
% Краевые условия, возникающие в точке $x_0 \in \left\{0,l\right\}$, соответствующие опертому, защемленному или
% свободному типу крепления балки, в указанном порядке имеют следующий вид:
% \[
% \begin{aligned}
% y(x_0) &= (EIy'')(x_0) = 0,
% \\
% y(x_0) &= y'(x_0) = 0,
% \\
% (EI y'')(x_0) &= \left(EI y''\right)'(x_0) = 0.
% \end{aligned}
% \]
%
%
%
% \par
% \textcolor{red}{
% В дальнейшем будем предполагать, что момент инерции сечения связан с распределением его по площади по закону
% \( I(x) = A_{\alpha_1}S^{\alpha_1}(x),\) где \({\alpha_1}, A_{\alpha_1}\) некоторые постоянные. Безусловно,
% особенный практический интерес представляют значения 1,2 и 3 параметра \(\alpha_1\).}
%
%
%
% \par
% Введем безразмерные переменные
% \begin{gather}
% x'=\frac{x}{l}, \;\;\; u'=\frac{u}{l}, \;\;\; p'=\frac{lS}{V},
% \notag
% \\
% \label{ChangeOfVariables2}
% \lambda'=\frac{\omega^2\rho V}{B'_{\alpha_1}}, \;\;\;
% B'_{\alpha_1}=\frac{EA_{\alpha_1}}{l^3}\left[\frac{V}{l}\right]^{\alpha_1}
% \end{gather}
% и перепишем уравнение \eqref{BeamVibrating:Dimensional}, опуская в дальнейшем штрихи:
% \begin{equation}
% \label{BeamVibrating:ClassicalSetting}
% \left(e u^\nu y''\right)'' = \lambda \rho u y,
% \qquad
% x \in I \triangleq (0, 1).
% \end{equation}
%
%
%
% \par
% В дальнейшем будем предпологать, что балка может быть закреплена одним из перечисленных ниже способов:
% \begin{enumerate}
%
%
%
% \item
% оперта на обоих концах (Рис.~\ref{fig:HingedBeam}):
% \begin{equation}
% y(0) = (e u^\nu y'')(0) = 0,
% \qquad
% y(1) = (e u^\nu y'')(1) = 0;
% \tag{BC${}_0^0$}
% \end{equation}
%
%
%
% \item
% жестко защемлена на обоих концах (Рис.~\ref{fig:ClampedBeam}):
% \begin{equation}
% y(0) = y'(0) = 0,
% \qquad
% y(1) = y'(1) = 0;
% \tag{BC${}_1^1$}
% \end{equation}
%
%
%
%
%
% \item
% жестко защемлена на одном из концов и свободна на другом (консольная балка, Рис.~\ref{fig:ClampedFreeBeam}):
%
%
%
% \begin{equation}
% y(0) = y'(0) = 0,
% \qquad
% (e u^\nu y'')(1) = (e u^\nu y'')'(1) = 0;
% \tag{BC${}_1^2$}
% \end{equation}
%
%
%
% \item
% жестко защемлена на одном из концов и шарнирно оперта на другом
% (Рис.~\ref{fig:ClampedHingedBeam}):
%
%
%
% \begin{equation}
% y(0) = y'(0) = 0,
% \qquad
% y(1) = (e u^\nu y'')(1) = 0.
% \tag{BC${}_1^0$}
% \end{equation}
%
%
%
%
%
% \end{enumerate}
%
%
%
%
%
% \begin{figure}[h]
% \label{fig:HingedBeam}
% 				\centering
% 				\begin{picture}(230,80)(0,0)
% 					\put( 15,15){\line( 1, 0){30}}
% 					\put( 15,15){\line( 1, 2){15}}
% 					\put( 45,15){\line( -1, 2){15}}
% 					\put( 185,15){\line( 1, 0){30}}
% 					\put( 185,15){\line( 1, 2){15}}
% 					\put( 215,15){\line( -1, 2){15}}
% 					\put( 5,45){\line( 1, 0){220}}
% 					\put( 5,75){\line( 1, 0){220}}
% 					\put( 5,45){\line( 0, 1){30}}
% 					\put( 225,45){\line( 0, 1){30}}
% 				\end{picture}
% 				\caption{Опертая на обоих концах балка.}
% \end{figure}
%
%
%
%
%
% \begin{figure}[h]
% \centering
% \begin{picture}(330,80)(0,0)
% \put( 55,35){\line( 1, 0){220}}
% 				\put( 55,65){\line( 1, 0){220}}
% 				\put( 55,15){\line( 0, 1){70}}
% 				\put( 275,15){\line( 0, 1){70}}
% 				\put( 35,15){\line( 5, 6){20}}
% 				\put( 35,35){\line( 5, 6){20}}
% 				\put( 35,55){\line( 5, 6){20}}
% 				\put( 275,25){\line( 5, 6){20}}
% 				\put( 275,45){\line( 5, 6){20}}
% 				\put( 275,65){\line( 5, 6){20}}
% 				\end{picture}
% 				\caption{Жестко защемленная на обоих концах балка.}
% \label{fig:ClampedBeam}
% \end{figure}
%
%
%
%
%
% \begin{figure}[h]
%     \centering
%     \begin{picture}(330,80)(0,0)
%     \put( 55,35){\line( 1, 0){220}}
%     \put( 55,65){\line( 1, 0){220}}
%     \put( 275,35){\line( 0, 1){30}}
%     \put( 55,15){\line( 0, 1){70}}
%     \put( 35,15){\line( 5, 6){20}}
%     \put( 35,35){\line( 5, 6){20}}
%     \put( 35,55){\line( 5, 6){20}}
%     \end{picture}
%     \caption{Консольная балка.}
%     \label{fig:ClampedFreeBeam}
% \end{figure}
%
%
%
%
%
%
%
%
%
%
% \begin{figure}[h]
%     \centering
%     \begin{picture}(330,80)(0,0)
%     \put( 245,05){\line( 1, 0){30}}
%     \put( 245,05){\line( 1, 2){15}}
%     \put( 275,05){\line( -1, 2){15}}
%     \put( 55,35){\line( 1, 0){220}}
%     \put( 55,65){\line( 1, 0){220}}
%     \put( 275,35){\line( 0, 1){30}}
%     \put( 55,15){\line( 0, 1){70}}
%     \put( 35,15){\line( 5, 6){20}}
%     \put( 35,35){\line( 5, 6){20}}
%     \put( 35,55){\line( 5, 6){20}}
%     \end{picture}
%     \caption{Защемленная на левом конце и опертая на правом балка.}
%     \label{fig:ClampedHingedBeam}
% \end{figure}
%
%
%
% \par
% Уравнение
% \eqref{BeamVibrating:ClassicalSetting}
% вместе с краевыми условиями
% $(\mathrm{BC}_i^j)$
% образует \textbf{\emph{краевую задачу на собственные значения}},
% которая состоит в определении пары $(\lambda, y)$,
% для которой справедливо уравнение
% \eqref{BeamVibrating:ClassicalSetting} и выполняются краевые условия $(\mathrm{BC}_i^j)$.
%
%
%
% Важно, что в паре $(\lambda, y)$
% функция $y$ должна быть нетривиальной, т.\,е. не равной тождественно нулю,
% т.\,к. в противном случае уравнение
% \eqref{BeamVibrating:ClassicalSetting} справедливо для любого $\lambda$ в силу его линейности.
%
%
%
% При соблюдении этой договоренности число $\lambda$
% называется \textbf{\emph{собственным значением}},
% а нетривиальное решение $y$ уравнения
% \eqref{BeamVibrating:ClassicalSetting}~---
% \textbf{\emph{собственной функцией (элементом)}}.
%
%
%
% Как следует из \S\,\ref{section:WeakSetting},
% задача
% \eqref{BeamVibrating:ClassicalSetting},
% $(\mathrm{BC}_i^j)$
% имеет счетное множество собственных чисел $\{ \lambda_k \}_{k \in \mathbb{N}}$,
% каждое из которых является положительным.
%
%
%
% В общей ситуации для краевой задачи на собственные значения
% собственному числу $\lambda$ может отвечать несколько линейно независимых
% собственных функций.
%
%
%
% Cовокупность всех собственных функций,
% отвечающих данному собственному числу $\lambda$,
% образует линейное пространство.
%
%
%
% Размерность этого пространства называется \textbf{\emph{кратностью}} собственного значения.
%
%
%
% Если кратность равна $1$, то собственное значение
% называется
% \textbf{\emph{простым}}.
%
%
%
% Из \S\,\ref{section:WeakSetting} следует,
% что
% все собственные значения $\lambda_k$ задачи
% \eqref{BeamVibrating:ClassicalSetting},
% $(\mathrm{BC}_i^j)$
% являются простыми.
%
%
%
%
%
% \par
% Собственные частоты $\omega_k$, где $k = 1, 2, \ldots$ образуют
% \textbf{\emph{спектр колебаний}} \cite{book:Timoshenko,book:Banichuk}.
%
%
%
% Если частоты прикладываемых к балке внешних возмущений находятся в интервале
% $(0, \omega_1)$,
% то не возникает негативных резонансных явлений.
%
%
%
% Следовательно,
% целесообразно
% расширить безрезонансный интервал частот $(0, \omega_1)$,
% чтобы обезопасить конструкцию от разрушения.
%
%
%
% Это приводит к задаче максимизации фундаментальной частоты $\omega_1$ свободных колебаний балки
% или соответственно первого собственного значения $\lambda_1$,
% так как в силу
% \eqref{ChangeOfVariables2}
% \[
% \lambda_k =
% \mathcal{O}(\omega_k^2).
% \]
%
%
%
% Решение этой задачи и является целью данной работы.
%
%
%
% Ее постановка рассматривается в
% \S\,\ref{section:OptimalControlProblem}.
