\chapter*{ЗАКЛЮЧЕНИЕ}
\addcontentsline{toc}{chapter}{ЗАКЛЮЧЕНИЕ}

В ходе проделанной работы был сформирован оригинальный набор размеченных данных, содержащих эмотивную лексику. На основе знаний полученных из рассмотренной литературы выбрано несколько архитектур моделей классификации. Был подробно разобран и обоснован их математический аппарат и написана реализация на языке Python c использованием библиотек машинного обучения и предобученых семантических моделей русского языка.

\bigskip
Был проведен сравнительный анализ результатов точности классификации метода опорных векторов (SVM), логистической регрессии (logistic regression) и случайного леса (random forest) в связке с распределенными представлениями слов: word2vec и ELMo.

\bigskip
Согласно результатам исследования лучшие метрики показала модель, основанная на случайном лесе с решающими деревьями и ELMo. Из этого следует вывод, что для задачи сентимент-анализа использование более сложных и тяжеловесных контекстуализированных эмбеддингов --- хорошее решение, т.к. они лучше обобщают сравнительно небольние тексты и позволяют классифиакору лучше разделить выборку. Но качество Macro F1 меры все равно не достигло высоких показателей. Возможно это связано с недостаточной величиной собранного набора данных в результате чего того  количества признаков, которые выделила модель, не хватило для качественной классификации.

\bigskip
Возможными направлениями для дальнейших исследований могут стать модели для выделения именнованых сущностей. Определение текстов относящихся именно к этим сущностиям и реализация сентимент-анализа с использованием информации от этих моделей в более комплексных архитектурах с механизмом внимания. Также увеличение набора данных поможет улучшить результаты.
































% \begin{enumerate}
% 	\item
% 	Исследована задача максимизации фундаментальной частоты свободных колебаний балки, решение которой позволяет находить оптимальные формы балки, при которых она наименее подвержена резонансу.
%
%
%
% 	Поэтому результаты данной работы могут найти практическое применение.
%
%
%
% 	\item
% 	В работе сформулирована итерационная процедура поиска седловой точки, позволяющая решить задачу максимизации фундаментальной частоты свободных колебаний балки.
%
%
%
% 	Этот подход к решению данной задачи ранее в литературе не применялся и поэтому является новым.
%
%
%
% 	Поэтому можно ожидать, что предложенный метод позволяет строить сходящиеся к оптимальному решению последовательности.
%
%
%
% 	\item
% 	Для этой итерационной процедуры был реализован численный метод в виде библиотеки на языке програмирования Python, а так же проведены вычислительные эксперементы, в ходе которых последовательность приближений демонстрировала сходимость.
%
%
%
% \end{enumerate}
%
%
