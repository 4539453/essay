\section{Численный метод решения задачи о поперечных колебаниях балки}
\label{section:NumericalMethod:EVP}

\begin{equation*}  
(Uy'')''=\lambda Vy
\end{equation*}
Полученное решение краевой задачи  \(y_{(n)}\) используем для определения следующего приближения безрезонансного интервала:
\begin{equation*}  
(Uy_{(n+1)}'')''=f=Vy_{(n)}
\end{equation*}
%
%
%
%
Вариационное представление
\eqref{FirstEigenvalue:VariationalPrinciple}
первого собственного значения $\lambda_1[u]$
определяем следущим \emph{\textbf{отношением Рэлея}}:
%
%
%
%
\begin{equation*}  
\lambda_1^{(n)}=\frac{\int_{0}^{1}Uy_{(n)}''^2dx}{\int_{0}^{1}Vy_{(n)}^2dx},
\end{equation*}
\begin{equation*}  
Uy_{(n)}''^2=\frac{1}{U}(Uy_{(n)}''^2)^2=\frac{1}{U}z^2,
\end{equation*}
\begin{equation}\label{eq9}   
\lambda_1^{(n)}=\frac{\int_{0}^{1}\frac{1}{U}z_{(n)}^2dx}{\int_{0}^{1}Vy_{(n)}^2dx}.
\end{equation}


Решение уравнения (\ref{eq9}) находим методом трапеций\\
\begin{equation}\label{eq10} 
\int_a^bf(x)\approx\frac{h}{2}(f(x_0)+2\sum\limits_{i=1}^{n-1} f(x_i) +f(x_n),
\end{equation}
\begin{equation*}  
h=\frac{b-a}{n}=1 \quad  x_i=a+ih \quad  i=0,1\ldots n,
\end{equation*}
\begin{center}  
\(i=0 \quad x_0+0*1\quad \Rightarrow\quad f(x_0)=Vy^2\bigg|_0\)\\
\(i=1 \quad x_1+1*1\quad \Rightarrow\quad f(x_1)=Vy^2\bigg|_1\)\\
\ldots\\
\ldots\\
\(i=n \quad x_n+1*n\quad \Rightarrow\quad f(x_n)=Vy^2\bigg|_n\)
\end{center}
\par
Далее заносим значения под знак суммы (\ref{eq10}) и находим все собственные значения $\lambda_1[u]$ описывающее резонансное поведение в процессе колебания балки, в этом случае следующая погрешность и является показательной
\begin{equation*}  
|\lambda_1^{(n+1)} + \lambda_1^{(n)}|<\varepsilon\approx 10^{10}.
\end{equation*}