\section{Обобщенная постановка краевой задачи на собственные значения. Свойства спектра}
\label{section:WeakSetting}

\par
В дальнейших рассмотрениях будем предполагать,
что
допустимое распределение площади поперечных сечений $u$ является элементом 
множества
\[
U = 
\left\{
a \in L^\infty(I) : \;
\alpha \leq a(x) \leq \beta
\right\}.
\]
Здесь $L^\infty(I)$ обозначает пространство всех измеримых существенно ограниченных функций,
определенных на отрезке $[0,1]$,
а параметры $\alpha$ и $\beta$ задают ограничение на возможные значения площади поперечного сечения вдоль пролета.
%
%
%
Предполагается, что
$0 < \alpha < \beta < +\infty$.
%
%
%
Также будем считать,
что
$e, \rho \in L^\infty(I)$ и, кроме того, справедливы неравенства
\[
e(x) \geq e_0,
\qquad
\rho(x) \geq \rho_0,
\]
где $e_0, \rho_0$~--- некоторые положительные постоянные.
%
%
%
\par
Обобщенная постановка краевой задачи на собственные значения
\eqref{BeamVibrating:ClassicalSetting}, $(\mathrm{BC}_i^j)$ имеет следующий вид:
найти пару $(\lambda, y) \in \mathbb{R} \times (V \setminus \{ \vartheta \})$ такую, что
\begin{equation}
\label{BeamVibrating:WeakSetting}
\int_0^1 eu^\nu y'' z'' \, dx
=
\lambda
\int_0^1 \rho u y z \, dx,
\qquad
z \in V.
\end{equation}
%
%
%
Здесь $V = V_i^j$ обозначает некоторое подпространство пространства Соболева
(конкретный вид которого будет представлен далее),
называемое \textbf{\emph{пространством состояний}},
отвечающее краевым условиям $(\mathrm{BC}_i^j)$,
которые, в свою очередь, соответствуют заданному типу крепления балки,
а
$\vartheta$~--- нулевой элемент этого пространства.
%
%
%
Заметим,
что
\eqref{BeamVibrating:WeakSetting} представляет собой интегральное тождество,
справедливое для всех элементов $z$ пространства $V$.
%
%
%
\par
Введем понятие пространства Соболева
\cite{book:OneDimVarProblems}.
%
%
%
Пусть $C^2([0, 1])$ обозначает пространство всех дважды непрерывно дифференцируемых функций,
определенных на отрезке $[0, 1]$.
%
%
%
Для элемента $y$ этого пространства введем норму по формуле
\begin{equation}
\label{SobolevNorm}
\lVert
y
\rVert_{H^2(I)}
=
\left(\int_0^1 \left( |y|^2 + |y'|^2 + |y''|^2 \right) \, dx\right)^{1/2}.
\end{equation}
%
%
%
\begin{definition}
Замыкание $C^2([0, 1])$ по норме
\eqref{SobolevNorm}
обозначается через $H^2(I)$ и называется \emph{\textbf{пространством Соболева}}.
\end{definition}
%
%
%
Как следует из
\cite{book:Fichera},
реализации пространства $V$,
отвечающие рассматриваемым типам закрепления балки,
имеют следующий вид:
\[
\begin{aligned}
V_0^0 &= \left\{
z \in H^2(I) : \;
z(0) = z(1) = 0
\right\},
\\
V_0^1 &= \left\{
z \in H^2(I) : \;
z(0) = z(1) = z'(1) = 0
\right\},
\\
V_1^1 &= \left\{
z \in H^2(I) : \;
z(0) = z'(0) = z(1) = z'(1) = 0
\right\},
\\
V_1^2 &= \left\{
z \in H^2(I) : \;
z(0) = z'(0) = 0
\right\}.
\end{aligned}
\]
%
%
%
\par
Через $L^2(I)$ будем обозначать пространство Лебега, состоящее из измеримых функций $y$,
квадраты которых являются суммируемыми,
т.\,е.
для которых
конечен следующий интеграл
\[
\int_0^1 y^2 \, dx.
\]
Напомним,
что норма элемента $y$ в пространстве $L^2(I)$
определяется формулой
\[
\lVert y \rVert_{L^2(I)} = \sqrt{\int_0^1 y^2 \, dx}.
\]
%
%
%
\par
Определим билинейные формы 
\[
\begin{aligned}
\mathcal{A}_u &: V \times V \to \mathbb{R},
\\
\mathcal{B}_u &: L^2(I) \times L^2(I) \to \mathbb{R},
\end{aligned}
\]
задаваемые формулами
\[
\begin{aligned}
\mathcal{A}_u(y, z) &= \int_0^1 eu^\nu y'' z'' \, dx,
\\
\mathcal{B}_u(y, z) &= \int_0^1 \rho u y z \, dx.
\end{aligned}
\]
%
%
%
Используя введенные обозначения,
перепишем
\eqref{BeamVibrating:WeakSetting}
в следующем виде:
\[
\mathcal{A}_u(y, z) = \lambda \mathcal{B}_u(y, z),
\qquad
z \in V.
\]
%
%
%
Для того чтобы выяснить вопрос о существовании собственных значений задачи \eqref{BeamVibrating:WeakSetting},
необходимо исследовать свойства введенных билинейных форм.
%
%
%
Пусть $H$~--- пространство Гильберта, $C : H \times H \to \mathbb{R}$~--- билинейная форма,
а
$\lVert \cdot \rVert_H$ обозначает норму в пространстве $H$.
%
%
%
Напомним следующие определения.
%
%
%
\begin{definition}
Билинейная форма $C$ называется \emph{\textbf{ограниченной}},
если найдется положительная постоянная $\hat{c}$
такая, что
для любых элементов $y, z \in H$ выполняется неравенство
\[
|C(y, z)|
\leq
\hat{c} \lVert y \rVert_H \lVert z \rVert_H.
\]
\end{definition}
%
%
%
\begin{definition}
Билинейная форма $C$ называется \emph{\textbf{коэрцитивной}},
если найдется положительная постоянная $\check{c}$
такая, что
для любого элемента $z \in H$ справедливо неравенство
\[
|C(z, z)|
\geq
\check{c} \lVert z \rVert_H^2.
\]
\end{definition}
%
%
%
\begin{proposition}
Билинейные формы $\mathcal{A}_u$, $\mathcal{B}_u$
являются ограниченными и коэрцитивными.
\end{proposition}
%
%
%
\begin{proof}
Напомним, что
\[
\rho \in L^\infty(I),
\qquad
u(x) \leq \beta.
\]
Тогда
\begin{equation}
\label{B:Boundedness}
|\mathcal{B}_u(y, z)|
=
\left|\int_0^1 \rho u y z \, dx
\right|
\leq
\int_0^1 \rho u |y z| \, dx
\leq
\beta
\lVert \rho \rVert_{L^\infty(I)}  \int_0^1 |y z| \, dx.
\end{equation}
Воспользовавшись известным неравенством Коши\nobreakdash-Буняковского
\cite{book:Kolmogorov,book:IP}
\[
\int_0^1 |y z| \, dx
\leq
\lVert y \rVert_{L^2(I)}
\lVert z \rVert_{L^2(I)},
\]
перепишем \eqref{B:Boundedness} в виде
\[
|\mathcal{B}_u(y, z)|
\leq
\beta
\lVert \rho \rVert_{L^\infty(I)}
\lVert y \rVert_{L^2(I)}
\lVert z \rVert_{L^2(I)}.
\]
Следовательно, билинейная форма $\mathcal{B}_u$ является ограниченной.
%
%
%
Аналогично получаем
\[
|\mathcal{A}_u(y, z)| = 
\left|
\int_0^1 eu^\nu y'' z'' \, dx
\right|
\leq
\int_0^1 eu^\nu |y'' z''| \, dx
\leq 
\beta^\nu
\lVert e \rVert_{L^\infty(I)}
\int_0^1 |y'' z''| \, dx.
\]
Снова применяя неравенство Коши\nobreakdash-Буняковского
теперь для выражения
\[
\int_0^1 |y'' z''| \, dx,
\]
заключаем,
что
\[
|\mathcal{A}_u(y, z)|
\leq
\beta^\nu
\lVert e \rVert_{L^\infty(I)}
\lVert y'' \rVert_{L^2(I)}
\lVert z'' \rVert_{L^2(I)}.
\]
Очевидно,
что
\[
\lVert y'' \rVert_{L^2(I)}
=
\sqrt{\int_0^1 |y''|^2 \, dx}
\leq
\left(\int_0^1 \left( |y|^2 + |y'|^2 + |y''|^2 \right) \, dx\right)^{1/2}
= \lVert y \rVert_{H^2(I)}.
\]
Таким образом,
\[
|\mathcal{A}_u(y, z)|
\leq
\beta^\nu
\lVert e \rVert_{L^\infty(I)}
\lVert y \rVert_{H^2(I)}
\lVert z \rVert_{H^2(I)},
\]
т.\,е.
билинейная форма
$\mathcal{A}_u$ является ограниченной.
%
%
%
Поскольку
\[
\rho(x) \geq \rho_0,
\qquad
u(x) \geq \alpha,
\]
то
\[
\mathcal{B}_u(z, z)
=
\int_0^1 \rho u z^2 \, dx
\geq
\alpha \rho_0
\int_0^1 z^2 \, dx
=
\alpha \rho_0 \lVert z \rVert_{L^2(I)}^2,
\]
что означает коэрцитивность билинейной формы $\mathcal{B}_u$,
так как
$\alpha, \rho_0 > 0$.
%
%
%
Коэрцитивность билинейной формы $\mathcal{A}_u$ для всех рассматриваемых реализаций пространства $V$
показана в
\cite{book:Fichera}.
\end{proof}
%
%
%
Ограниченность и коэрцитивность билинейных форм
$\mathcal{A}_u$ и $\mathcal{B}_u$ означает справедливость следующего известного утверждения.
%
%
%
\begin{theorem}
Задача
\eqref{BeamVibrating:WeakSetting}
имеет счетное множество собственных чисел
\[
0 < \lambda_1[u] \leq \lambda_2[u] \leq \ldots
\leq 
\lambda_k[u] \leq \ldots,
\qquad
\lim_{k \to \infty} \lambda_k[u] = +\infty
\]
и соответствующую последовательность собственных функций $\{ y_k \}$,
образующую ортогональный базис в $V$ и $L^2(I)$.
%
%
%
Кроме того, справедлив следующий вариационный принцип:
\begin{equation}
\label{FirstEigenvalue:VariationalPrinciple}
\lambda_1[u]
=
\min_{y \in V \setminus \{ \vartheta \}} \frac{\mathcal{A}_u(y,y)}{\mathcal{B}_u(y,y)}
=
\frac{\mathcal{A}_u(y_1,y_1)}{\mathcal{B}_u(y_1,y_1)}.
\end{equation}
%
%
%
Наконец, каждое собственное значение
$\lambda_k[u]$ имеет конечную кратность.
\end{theorem}
%
%
%
\noindent\emph{Доказательство} теоремы приводится в \cite{book:Litvinov}.
%
%
%
\par
Отметим,
что в литературе по механике (см., например,
\cite{book:Banichuk}) вариационное представление
\eqref{FirstEigenvalue:VariationalPrinciple}
первого собственного значения $\lambda_1[u]$
принято называть \emph{\textbf{вариационным принципом Рэлея}}.
%
%
%
Вместе с тем выражение
\[
\frac{\mathcal{A}_u(\cdot,\cdot)}{\mathcal{B}_u(\cdot,\cdot)}
\]
именуют \emph{\textbf{отношением Рэлея}}.
%
%
%
Вариационный принцип Рэлея означает,
что для любого ненулевого элемента $z$ из пространства состояний
$V$ выполняется неравенство
\[
\lambda_1[u]
=
\frac{\mathcal{A}_u(y_1,y_1)}{\mathcal{B}_u(y_1,y_1)}
\leq
\frac{\mathcal{A}_u(z,z)}{\mathcal{B}_u(z,z)},
\]
которое также можно записать в следующем виде:
\[
\lambda_1[u]
=
\frac
{\int_0^1 eu^\nu |y_1''|^2 \, dx}
{\int_0^1 \rho u y_1^2 \, dx}
\leq
\frac
{\int_0^1 eu^\nu |z''|^2 \, dx}
{\int_0^1 \rho u z^2 \, dx}.
\]
%
%
%
%
%
\par
Что касается кратности собственных значений 
задачи
\eqref{BeamVibrating:WeakSetting},
то
в действительности, как следует из
\cite{article:FourthOrderOscillations},
справедлива следующая
\begin{theorem}
Все собственные значения задачи
\eqref{BeamVibrating:WeakSetting}
являются простыми,
т.\,е.
\[
0 < \lambda_1[u] < \lambda_2[u] < \ldots
<
\lambda_k[u] < \ldots.
\]
\end{theorem}
%
%
%
В дальнейшем нам также понадобятся свойства собственных функций
задачи
\eqref{BeamVibrating:WeakSetting}
(см. \cite{article:FourthOrderOscillations}).
\begin{theorem}
Собственная функция,
соответствующая первому собственному значению задачи
\eqref{BeamVibrating:WeakSetting},
не меняет знака на интервале $I$.
\end{theorem}