\section{Поиск седловой точки}


\par
Оптимальное решение $w$ и соответствующая ему собственная функция $y_1$ образуют седловую точку функционала 
\begin{equation}
F(u,z) = \frac{\int_{0}^{1}eu^{\nu}z''^2  \, dx}{\int_{0}^{1}\rho uz^2 \, dx},
\end{equation}
т.\,е. выполняются неравенства
\begin{equation}
F(u,y_1) \leq F(w,y_1) \leq F(w,z),
\end{equation}
где $u$ и $z$ пробегают допустимые множества.
%
\bigskip
\par
Седловую точку ищем с помощью следущей процедуры: 
%
\par 
\begin{enumerate}
	\item задается начальное приближение $u_0 \in U$;
	%
	\item  для заданного $u_n$ решается задача на собственные значения для определения собственной пары ($\lambda_1[u_n], y_1[u_n]$);
	%Это эквивалентно решению задачи на собственные значения.
	
	\par
	%
	\item находится новое приближение $u_{n+1}$, для которого
	\begin{equation}
	F(u_{n+1},y_1[u_n]) = \max_{u \in U} F(u,y_1[u_n]);
	\end{equation}
	
	\item если 
	\begin{equation}
	|\lambda_1[u_{n+1}] - \lambda_1[u_{n}]| < \varepsilon,
	\end{equation}
	где $\varepsilon$~--- заданная  точность, то итерационная процедура завершается, а $u_{n+1}$ выступает в качестве оптимального распределения.
	\bigskip
	\par
	В противном случае переходим к шагу 2 с $n=n+1$.
\end{enumerate}

\bigskip
\par
После определения собственной пары составляем уравнение максимизации колебательного процесса упругой балки, для определения нового оптимального приближения, с последующим определением новой собственной пары, для продолжения итерационного процесса
\begin{equation}
\label{MaximizingVibrating} 
e\xi y''^2_n - \lambda\rho\xi y^2_n + \mu\xi \rightarrow \max,
\end{equation}

\begin{equation*}
\begin{aligned}
\int_0^1{eU_{n+1}y''^2_ndx} - \lambda_n\int_0^1{\rho U_{n+1}y^2_ndx}\geqslant	\\
\geqslant \int_0^1{eU_ny''^2_ndx} - \lambda_n\int_0^1{\rho U_ny^2_ndx}\\
\end{aligned}
\end{equation*}

\begin{equation*} 
e(x)\xi y''^2_n - \lambda\rho(x)\xi y^2_n + \mu\xi \rightarrow \max.
\end{equation*}
%
%
%
Подставляем $\mu$  и для каждого x, решим задачу получив $\xi(x)$,
%
%
%
\begin{equation*} 
\xi(e(x) y''^2_n - \lambda\rho(x)\xi y^2_n + \mu) \rightarrow \max,
\end{equation*}
$\mu$ надо подобрать так, чтоб интеграл $\int_0^1{\xi(x)dx} = 1$ был равен единице, вводим следующую функцию: 
\begin{equation*} 
f(\mu)=\int_0^1{\xi_{\mu}(x)dx} - 1,
\end{equation*}
\begin{equation*} 
\xi \in [\alpha,\beta] \qquad \xi(x) \rightarrow \max,
\end{equation*}
%
%
%
где $\alpha , \beta$~--- ограничения на площадь, $\alpha= u_{min}$, $\beta= u_{max}$ 
%
%
%
\begin{center}
$$ (\lambda_n,y_n) \rightarrow \xi(x)= U_{n+1},$$
$$ (\lambda_{n+1},y_{n+1}) \rightarrow \xi(x)= U_{n+2},$$
\ldots \\
\ldots \\
$$(\lambda_{n+k-1},y_{n+k-1}) \rightarrow \xi(x)= U_{n+k};$$
\end{center}
%
%
%
%
Для решения оптимизационной поточечной задачи, использем метод секущих Ньютона (не требующий производную функции) находим ноль функции, и задача \eqref{MaximizingVibrating} принимает следущий вид
%
%
%
%
\begin{equation*} 
f(\mu)=\int_0^1{\xi_{\mu}(\frac{ez^2_n}{u^2\rho^2}+\lambda\rho y^2_n+\mu)dx} - 1
\end{equation*}
%
%
метод секущих для следующей итерации $x_n$
%
%
\begin{equation*} 
x_n=x_{n-1} - f(x_n -1) \frac{x_{n-1}-x_{n-2}}{f(x_{n-1})-f(x_{n-2})}=\frac{x_{n-2}f(x_n-1)-x_{n-1}f(x_{n-2})}{f(x_{n-1}-f(x_{n-2}))}.
\end{equation*}
Для первого выполнения итерации берем  $x_{n-1}$~--- значения первичной собственной пары $x_{n-2} = 0$. Задача на собственные значения решается многократно, останавливается когда $|\lambda_{n+1}-\lambda_n| <\varepsilon\approx 10^{5}$

%О том как ищешь седловую точку.

%Поточечная задача оптимизации от $\xi$ и т.д.

%Как решаешь эту задачу.
