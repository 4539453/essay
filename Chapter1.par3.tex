\section{Метод последовательных приближений определения собственной пары}
\label{section:ApproximationsMethod}


\begin{definition}
Первое собственное значение $\lambda_1[u]$ краевой задачи
\eqref{BeamVibrating:WeakSetting},
а также соответствующую ему собственную функцию $y$
будем называть \emph{\textbf{собственной парой}} этой задачи и записывать через $(\lambda_1[u], y)$.
\end{definition}
%
%
%
\par
В дальнейшем при численном решении задачи оптимального проектирования колеблющейся балки
(которая рассматривается в Главе~\ref{chapter:OptimalBeamDesign})
нам понадобится итерационный метод для приближенного определения собственной пары задачи
\eqref{BeamVibrating:WeakSetting}.
%
%
%
С этой целью будем применять стандартный \emph{\textbf{метод последовательных приближений}},
который состоит в построении последовательности
приближений $\{ (\lambda_i, y_i) \}$ собственной пары,
элементы которой определяются из следующих формул:
\[
(eu^\nu y_{i + i}'')'' = \rho u y_i,
\qquad
\lambda_i = \Lambda(u, y_i),
\]
где в качестве начального приближения $y_0$ берется дважды непрерывно дифференцируемая функция,
удовлетовряющая заданным краевым условиям,
а
\[
\Lambda(u, z)
\triangleq
\frac{\mathcal{A}_u(z,z)}{\mathcal{B}_u(z,z)}
=
\frac
{\int_0^1 eu^\nu |z''|^2 \, dx}
{\int_0^1 \rho u z^2 \, dx}.
\]
%
%
%
Сходимость метода последовательных приближений
обосновывается в
\cite{book:Collatz}.
